% \begin{meta-comment}
%
% $Id$
%
% Allow @-commands
%
% (c) 1995 Mark Wooding
%
%----- Revision history -----------------------------------------------------
%
% $Log$
% Revision 1.1  2000-07-13 09:10:20  michael
% + Initial import
%
% Revision 1.1  1998/09/21 10:18:06  michael
% Initial implementation
%
% Revision 1.3  1996/11/19 20:46:55  mdw
% Entered into RCS
%
%
% \end{meta-comment}
%
% \begin{meta-comment} <general public licence>
%%
%% at package -- support for `@' commands'
%% Copyright (c) 1996 Mark Wooding
%%
%% This program is free software; you can redistribute it and/or modify
%% it under the terms of the GNU General Public License as published by
%% the Free Software Foundation; either version 2 of the License, or
%% (at your option) any later version.
%%
%% This program is distributed in the hope that it will be useful,
%% but WITHOUT ANY WARRANTY; without even the implied warranty of
%% MERCHANTABILITY or FITNESS FOR A PARTICULAR PURPOSE.  See the
%% GNU General Public License for more details.
%%
%% You should have received a copy of the GNU General Public License
%% along with this program; if not, write to the Free Software
%% Foundation, Inc., 675 Mass Ave, Cambridge, MA 02139, USA.
%%
% \end{meta-comment}
%
% \begin{meta-comment} <Package preamble>
%<+package>\NeedsTeXFormat{LaTeX2e}
%<+package>\ProvidesPackage{at}
%<+package>                [1996/05/02 1.3 @-command support (MDW)]
% \end{meta-comment}
%
% \CheckSum{355}
%% \CharacterTable
%%  {Upper-case    \A\B\C\D\E\F\G\H\I\J\K\L\M\N\O\P\Q\R\S\T\U\V\W\X\Y\Z
%%   Lower-case    \a\b\c\d\e\f\g\h\i\j\k\l\m\n\o\p\q\r\s\t\u\v\w\x\y\z
%%   Digits        \0\1\2\3\4\5\6\7\8\9
%%   Exclamation   \!     Double quote  \"     Hash (number) \#
%%   Dollar        \$     Percent       \%     Ampersand     \&
%%   Acute accent  \'     Left paren    \(     Right paren   \)
%%   Asterisk      \*     Plus          \+     Comma         \,
%%   Minus         \-     Point         \.     Solidus       \/
%%   Colon         \:     Semicolon     \;     Less than     \<
%%   Equals        \=     Greater than  \>     Question mark \?
%%   Commercial at \@     Left bracket  \[     Backslash     \\
%%   Right bracket \]     Circumflex    \^     Underscore    \_
%%   Grave accent  \`     Left brace    \{     Vertical bar  \|
%%   Right brace   \}     Tilde         \~}
%%
%
% \begin{meta-comment} <driver>
%
%<*driver>
%
% $Id$
%
% Installer for the mdwtools packages
%
% (c) 1996 Mark Wooding
%

%----- Revision history -----------------------------------------------------
%
% $Log$
% Revision 1.1  2000-07-13 09:10:21  michael
% + Initial import
%
% Revision 1.1  1998/09/21 10:19:01  michael
% Initial implementation
%
% Revision 1.3  1996/11/19 20:57:26  mdw
% Entered into RCS
%

% --- Licence note ---
%
% mdwtools installer
% Copyright (c) 1996 Mark Wooding
%
% This program is free software; you can redistribute it and/or modify
% it under the terms of the GNU General Public License as published by
% the Free Software Foundation; either version 2 of the License, or
% (at your option) any later version.
%
% This program is distributed in the hope that it will be useful,
% but WITHOUT ANY WARRANTY; without even the implied warranty of
% MERCHANTABILITY or FITNESS FOR A PARTICULAR PURPOSE.  See the
% GNU General Public License for more details.
%
% You should have received a copy of the GNU General Public License
% along with this program; if not, write to the Free Software
% Foundation, Inc., 675 Mass Ave, Cambridge, MA 02139, USA.

% --- Sort out how to do all this ---

\def\batchfile{mdwtools.ins}
\input docstrip
\keepsilent

\preamble

IMPORTANT NOTICE
\endpreamble

{ % --- This is a group so that docstrip \reads it all in one go ---

  \ifx\generate\mdwxxnotdef
    \gdef\mdwgen#1{#1}
    \gdef\mdwf#1#2{\generateFile{#1}{n}{#2}}
    \gdef\needed#1{}
  \else
    \global\let\mdwgen\generate
    \global\def\mdwf{\file}
    \global\askforoverwritefalse
  \fi

}

\mdwgen{\mdwf {at.sty}		{\from {at.dtx}	      {package}}
	\mdwf {mdwlist.sty}	{\from {mdwlist.dtx}  {package}}
	\mdwf {mdwtab.sty}	{\from {mdwtab.dtx}   {mdwtab}
				 \needed{syntax.dtx}
				 \from {footnote.dtx} {macro}
				 \from {doafter.dtx}  {macro}}
	\mdwf {syntax.sty}	{\from {syntax.dtx}   {package}
				 \from {doafter.dtx}  {macro}}
	\mdwf {mathenv.sty}	{\from {mdwtab.dtx}   {mathenv}}
	\mdwf {mdwmath.sty}	{\from {mdwmath.dtx}  {package}}
	\mdwf {sverb.sty}	{\from {sverb.dtx}    {package}}
	\mdwf {footnote.sty}	{\from {footnote.dtx} {package}}
	\mdwf {doafter.sty}	{\from {doafter.dtx}  {package,latex2e}}
	\mdwf {doafter.tex}	{\from {doafter.dtx}  {package,plain}}
	\mdwf {cmtt.sty}	{\from {cmtt.dtx}     {sty}}
	\mdwf {mTTenc.def}	{\from {cmtt.dtx}     {def}}
	\mdwf {mTTcmtt.fd}	{\from {cmtt.dtx}     {fd}}
}

\Msg{Done!}

\describespackage{at}
\aton
\atlet p=\package
\atdef at{\package{at}}
\atdef={\mbox{-}}
\atdef-{@@@=}
\atlet.=\syntax
\mdwdoc
%</driver>
%
% \end{meta-comment}
%
% \section{User guide}
%
% The @at\ package is an attempt to remove a lot of tedious typing that
% ends up in \LaTeX\ documents, by expanding the number of short command
% names available.  The new command names begin with the `|@|' character,
% rather than the conventional `|\|', so you can tell them apart.
%
% The package provides some general commands for defining @-commands, and
% then uses them to define some fairly simple ones which will be useful to
% most people.
%
% The rules for @-command names aren't terribly complex:
% \begin{itemize}
% \item If the first character of the name is a letter, then the command name
%       consists of all characters up to, but not including, the first
%       nonletter.  Spaces following the command name are ignored.
% \item If the first character of the name is a backslash, then the @-command
%       name consists of the control sequence introduced by the backslash.
% \item Otherwise, the command name consists only of that first character.
%       Spaces following the name are not ignored, unless that character
%       was itself a space character.
% \end{itemize}
%
% Usually, digits are not considered to be letters.  However, the
% \package{at} package will consider digits to be letters if you give it the
% \textsf{digits} option in the |\usepackage| command.  (Note that this
% only affects the \package{at} package; it won't change the characters
% allowed in normal command names.)
%
% \DescribeMacro{\atallowdigits}
% \DescribeMacro{\atdisallowdigits}
% You can enable and disable digits being considered as letters dynamically.
% The |\atallowdigits| command allows digits to be used as letters;
% |\atdisallowdigits| prevents this.  Both declarations follow \LaTeX's
% usual scoping rules.  Both of these commands have corresponding
% environments with the same names (without the leading `|\|', obviously).
%
% \subsection{Defining @-commands}
%
% \DescribeMacro{\newatcommand}
% \DescribeMacro{\renewatcommand}
% The |\newatcommand| command will define a new @-command using a syntax
% similar to |\newcommand|.  For example, you could define
% \begin{listing}
%\newatcommand c[1]{\chapter{#1}}
% \end{listing}
% to make @.{"@c{"<name>"}"} equivalent to @.{"\\chapter{"<name>"}"}.
%
% A |\renewatcommand| is also provided to redefine existing commands, should
% the need arise.
%
% \DescribeMacro{\atdef}
% For \TeX\ hackers, the |\atdef| command defines @-commands using a syntax
% similar to \TeX's built-in |\def|.  
%
% As an example, the following command makes @.{"@/"<text>"/"} write its
% argument \<text> in italics:
% \begin{listing}
%\atdef/#1/{\textit{#1}}
% \end{listing}
% The real implementation of the |@/|\dots|/| command is a bit more
% complex, and is given in the next section.
%
% You can use all of \TeX's features for defining the syntax of your
% command.  (See chapter~20 of @/The \TeX book/ for more details.)
%
% \DescribeMacro{\atlet}
% Since |\atdef| is provided to behave similarly to |\def|, @at\ provides
% |\atlet| which works similarly to |\let|.  For example you can say
% \begin{listing}
%\atlet!=\index
% \end{listing}
% to allow the short |@!| to behave exactly like |\index|.
%
% Note that all commands defined using these commands are robust even if you
% use fragile commands in their definitions.  Unless you start doing very
% strange things, @-commands never need |\protect|ing.
%
% \subsection{Predefined @-commands}
%
% A small number of hopefully useful commands are provided by default.
% These are described in the table below:
%
% \bigskip \begin{center} \begin{tabular}{lp{3in}}                    \hline
% \bf Command        & \bf Meaning                                 \\ \hline
% @.{"@@"}           & Typesets an `@@' character.                 \\
% @.{"@/"<text>"/"}  & In text (LR or paragraph) mode, typesets its
%                      argument emphasised.  In maths mode, it
%                      always chooses italics.                     \\
% @.{"@*"<text>"*"}  & Typesets its argument \<text> in bold.      \\
% @.{"@i{"<text>"}"} & Equivalent to `@.{"\\index{"<text>"}"}'.    \\
% @.{"@I{"<text>"}"} & As for |@i|, but also writes its argument
%                      to the document.                            \\ \hline
% \end{tabular} \end{center} \bigskip
%
% Package writers should not rely on any predefined @-commands -- they're
% provided for users, and users should be able to redefine them without
% fear of messing anything up.  (This includes the `standard' commands
% provided by the @at\ package, by the way.  They're provided in the vague
% hope that they might be useful, and as examples.)
%
% \implementation
%
% \section{Implementation}
%
%    \begin{macrocode}
%<*package>
%    \end{macrocode}
%
% \subsection{Options handling}
%
% We need a switch to say whether digits should be allowed.  Since this
% is a user thing, I'll avoid |\newif| and just define the thing by hand.
%
%    \begin{macrocode}
\def\atallowdigits{\let\ifat@digits\iftrue}
\def\atdisallowdigits{\let\ifat@digits\iffalse}
%    \end{macrocode}
%
% Now define the options.
%
%    \begin{macrocode}
\DeclareOption{digits}{\atallowdigits}
\DeclareOption{nodigits}{\atdisallowdigits}
\ExecuteOptions{nodigits}
\ProcessOptions
%    \end{macrocode}
%
% \subsection{How the commands work}
%
% Obviously we make the `@@' character active.  It inspects the next
% character (or argument, actually -- it can be enclosed in braces for
% longer commands, although this is a bit futile), and builds the command
% name from that.
%
% The |\at| command is equivalent to the active `@@' character always.
%
%
% \subsection{Converting command names}
%
% We need to be able to read an @-command name, and convert it to a normal
% \TeX\ control sequence.  First, we declare some control sequences for
% braces, which we need later.
%
%    \begin{macrocode}
\begingroup
\catcode`\<1
\catcode`\>2
\catcode`\{12
\catcode`\}12
\gdef\at@lb<{>
\gdef\at@rb<}>
\gdef\at@spc< >
\endgroup
%    \end{macrocode}
%
% I'll set up some helper routines now, to help me read the command
% names.  The way this works is that we |\futurelet| the token into
% |\@let@token|.  These routines will then sort out what to do next.
%
% \begin{macro}{\at@test}
%
% Given an |\if|\dots\ test, does its first or second argument.
%
%    \begin{macrocode}
\def\at@test#1\then{%
  #1\expandafter\@firstoftwo\else\expandafter\@secondoftwo\fi%
}
%    \end{macrocode}
%
% \end{macro}
%
% \begin{macro}{\at@ifcat}
%
% Checks the category code of the current character.  If it matches the
% argument, it does its second argument, otherwise it does the third.
%
%    \begin{macrocode}
\def\at@ifcat#1{\at@test\ifcat#1\noexpand\@let@token\then}
%    \end{macrocode}
%
% \end{macro}
%
% \begin{macro}{\at@ifletter}
%
% This routine tests the token to see if it's a letter, and if so adds
% it to the token list and does the first argument; otherwise it does the
% second argument.  It accepts digits as letters if the switch is turned
% on.
%
% There's some fun later, so I'll describe this slowly.  First, we compare
% the category code to a letter, and if we have a match, we know we're done;
% we need to pick up the letter as an argument.  If the catcode is `other',
% we must compare with numbers to see if it's in range.
%
%    \begin{macrocode}
\def\at@ifletter#1#2{%
  \at@ifcat x%
    {\at@ifletter@ii{#1}}%
    {\at@ifcat 0%
      {\at@ifletter@i{#1}{#2}}%
      {#2}%
    }% 
}
%    \end{macrocode}
%
% Right.  It's `other' (so it's safe to handle as a macro argument) and we
% need to know if it's a digit.  This is a little tricky: I use |\if| to
% compare two characters.  The first character is~`1' or~`0' depending on the
% `digit' switch; the second is~`1' or~`x' depending on whether it's actually
% a digit.  They'll only match if everything's worked out OK.
%
%    \begin{macrocode}
\def\at@ifletter@i#1#2#3{%
  \at@test\if%
    \ifat@digits1\else0\fi%
    \ifnum`#3<`0x\else\ifnum`#3>`9x\else1\fi\fi%
  \then%
    {\at@ifletter@ii{#1}{#3}}%
    {#2#3}%
}
%    \end{macrocode}
%
% Right; we have the character, so add it to the list and carry on.
%
%    \begin{macrocode}
\def\at@ifletter@ii#1#2{\toks@\expandafter{\the\toks@#2}#1}
%    \end{macrocode}
%
% \end{macro}
%
% Now we define the command name reading routines.  We have @/almost/ the
% same behaviour as \TeX, although we can't support `|%|' characters for
% reasons to do with \TeX's tokenising algorithm.
%
% \begin{macro}{\at@read@name}
%
% The routine which actually reads the command name works as follows:
% \begin{enumerate}
% \item Have a peek at the next character.  If it's a left or right brace,
%       then use the appropriate character.
% \item If the character is not a letter, just use the character (or whole
%       control sequence.
% \item Finally, if it's a letter, keep reading letters until we find one
%       that wasn't.
% \end{enumerate}
%
% First, we do some setting up and read the first character
%
%    \begin{macrocode}
\def\at@read@name#1{%
  \let\at@next=#1%
  \toks@{}%
  \futurelet\@let@token\at@rn@i%
}
%    \end{macrocode}
%
% Next, sort out what to do, based on the category code.
%
%    \begin{macrocode}
\def\at@rn@i{%
  \def\@tempa{\afterassignment\at@rn@iv\let\@let@token= }%
  \at@ifletter%
    {\futurelet\@let@token\at@rn@iii}%
    {\at@ifcat\bgroup%
      {\toks@\expandafter{\at@lb}\@tempa}%
      {\at@ifcat\egroup%
        {\toks@\expandafter{\at@rb}\@tempa}%
        {\at@ifcat\at@spc%
          {\toks@{ }\@tempa}%
          {\at@rn@ii}%
        }%
      }%
    }%
}
%    \end{macrocode}
%
% Most types of tokens can be fiddled using |\string|.
%
%    \begin{macrocode}
\def\at@rn@ii#1{%
  \toks@\expandafter{\string#1}%
  \at@rn@iv%
}
%    \end{macrocode}
%
% We've found a letter, so we should check for another one.
%
%    \begin{macrocode}
\def\at@rn@iii{%
  \at@ifletter%
    {\futurelet\@let@token\at@rn@iii}%
    {\@ifnextchar.\at@rn@iv\at@rn@iv}%
}
%    \end{macrocode}
%
% Finally, we need to pass the real string, as an argument, to the
% macro.  We make |\@let@token| relax, since it might be something which will
% upset \TeX\ later, e.g., a |#| character.
%
%    \begin{macrocode}
\def\at@rn@iv{%
  \let\@let@token\relax%
  \expandafter\at@next\csname at.\the\toks@\endcsname%
}
%    \end{macrocode}
%
% \end{macro}
%
% \begin{macro}{\at@cmdname}
%
% Given a control sequence, work out which @-command it came from.
%
%    \begin{macrocode}
\def\at@cmdname#1{\expandafter\at@cmdname@i\string#1\@@foo}
%    \end{macrocode}
%
% Now extract the trailing bits.
%
%    \begin{macrocode}
\def\at@cmdname@i#1.#2\@@foo{#2}
%    \end{macrocode}
%
% \end{macro}
%
% \begin{macro}{\at@decode}
%
% The |\at@decode| macro takes an extracted @-command name, and tries to
% execute the correct control sequence derived from it.
%
%    \begin{macrocode}
\def\at@decode#1{%
  \at@test\ifx#1\relax\then{%
    \PackageError{at}{Unknown @-command `@\at@cmdname#1'}{%
      The @-command you typed wasn't recognised, so I've ignored it.
    }%
  }{%
    #1%
  }%
}
%    \end{macrocode}
%
% \end{macro}
%
% \begin{macro}{\@at}
%
% We'd like a measure of compatibility with @p{amsmath}.  The @-commands
% provided by @p{amsmath} work only in maths mode, so this gives us a way of
% distinguishing.  If the control sequence |\Iat| is defined, and we're in
% maths mode, we'll call that instead of doing our own thing.
%
%    \begin{macrocode}
\def\@at{%
  \def\@tempa{\at@read@name\at@decode}%
  \ifmmode\ifx\Iat\not@@defined\else%
    \let\@tempa\Iat%
  \fi\fi%
  \@tempa%
}
%    \end{macrocode}
%
% \end{macro}
%
%
% \subsection{Defining new commands}
%
% \begin{macro}{\at@buildcmd}
%
% First, we define a command to build these other commands:
%
%    \begin{macrocode}
\def\at@buildcmd#1#2{%
  \expandafter\def\csname\expandafter
    \@gobble\string#1@decode\endcsname##1{#2##1}%
  \edef#1{%
    \noexpand\at@read@name%
    \expandafter\noexpand%
      \csname\expandafter\@gobble\string#1@decode\endcsname%
  }%
}
%    \end{macrocode}
%
% \end{macro}
%
% \begin{macro}{\newatcommand}
% \begin{macro}{\renewatcommand}
% \begin{macro}{\provideatcommand}
% \begin{macro}{\atdef}
% \begin{macro}{\atshow}
%
% Now we define the various operations on @-commands.
%
%    \begin{macrocode}
\at@buildcmd\newatcommand\newcommand
\at@buildcmd\renewatcommand\renewcommand
\at@buildcmd\provideatcommand\providecommand
\at@buildcmd\atdef\def
\at@buildcmd\atshow\show
%    \end{macrocode}
%
% \end{macro}
% \end{macro}
% \end{macro}
% \end{macro}
% \end{macro}
%
% \begin{macro}{\atlet}
%
% |\atlet| is rather harder than the others, because we want to allow people
% to say things like @.{"\\atlet"<name>"=@"<name>}.  The following hacking
% does the trick.  I'm trying very hard to duplicate |\let|'s behaviour with
% respect to space tokens here, to avoid any surprises, although there
% probably will be some differences.  In particular, |\afterassignment|
% won't work in any sensible way.
%
% First, we read the name of the @-command we're defining.  We also open
% a group, to stop messing other people up, and make `@@' into an `other'
% token, so that it doesn't irritatingly look like its meaning as a control
% sequence.
%
%    \begin{macrocode}
\def\atlet{%
  \begingroup%
  \@makeother\@%
  \at@read@name\atlet@i%
}
%    \end{macrocode}
%
% Put the name into a scratch macro for later use.  Now see if there's an
% equals sign up ahead.  If not, this will gobble any spaces in between the
% @-command name and the argument.
%
%    \begin{macrocode}
\def\atlet@i#1{%
  \def\at@temp{#1}%
  \@ifnextchar=\atlet@ii{\atlet@ii=}%
}
%    \end{macrocode}
%
% Now we gobble the equals sign (whatever catcode it is), and peek at the
% next token up ahead using |\let| with no following space.
%
%    \begin{macrocode}
\def\atlet@ii#1{\afterassignment\atlet@iii\global\let\at@gnext=}
%    \end{macrocode}
%
% The control sequence |\at@gnext| is now |\let| to be whatever we want the
% @-command to be, unless it's picked up an `@@' sign.  If it has, we've
% eaten the |@| token, so just read the name and pass it on.  Otherwise,
% we can |\let| the @-command directly to |\at@gnext|.  There's some
% nastiness here to make |\the\toks@| expand before we close the group and
% restore its previous definition.
%
%    \begin{macrocode}
\def\atlet@iii{%
  \if @\noexpand\at@gnext%
    \expandafter\at@read@name\expandafter\atlet@iv%
  \else%
    \expandafter\endgroup%
    \expandafter\let\at@temp= \at@gnext%
  \fi%
}
%    \end{macrocode}
%
% We've read the source @-command name, so just copy the definitions over.
%
%    \begin{macrocode}
\def\atlet@iv#1{%
  \expandafter\endgroup%
  \expandafter\let\at@temp=#1%
}
%    \end{macrocode}
%
% \end{macro}
%
%
% \subsection{Robustness of @-commands}
%
% We want all @-commands to be robust.  We could leave them all being
% fragile, although making robust @-commands would then be almost impossible.
% There are two problems which we must face:
%
% \begin{itemize}
%
% \item The `|\@at|' command which scans the @-command name is (very)
%       fragile.  I could have used |\DeclareRobustCommand| for it (and in
%       fact I did in an earlier version), but that doesn't help the other
%       problem at all.
%
% \item The `name' of the @-command may contain active characters or control
%       sequences, which will be expanded at the wrong time unless we do
%       something about it now.
%
% \end{itemize}
%
% We must also be careful not to introduce extra space characters into any
% files written, because spaces are significant in @-commands.  Finally,
% we have a minor problem in that most auxiliary files are read in with
% the `@@' character set to be a letter.
%
% \begin{macro}{\at}
%
% Following the example of \LaTeX's `short' command handling, we'll define
% |\at| to decide what to do depending on what |\protect| looks like.  If
% we're typesetting, we just call |\@at| (above) and expect it to cope.
% Otherwise we call |\at@protect|, which scoops up the |\fi| and the |\@at|,
% and inserts other magic.
%
%    \begin{macrocode}
\def\at{\ifx\protect\@typeset@protect\else\at@protect\fi\@at}
%    \end{macrocode}
%
% \end{macro}
%
% \begin{macro}{\at@protect}
%
% Since we gobbled the |\fi| from the above, we must put that back.  We then 
% need to do things which are more complicated.  If |\protect| is behaving 
% like |\string|, then we do one sort of protection.  Otherwise, we assume
% that |\protect| is being like |\noexpand|.
%
%    \begin{macrocode}
\def\at@protect\fi#1{%
  \fi%
  \ifx\protect\string%
    \expandafter\at@protect@string%
  \else%
    \expandafter\at@protect@noexpand%
  \fi%
}
%    \end{macrocode}
%
% \end{macro}
%
% \begin{macro}{\at@protect@string}
%
% When |\protect| is |\string|, we don't need to be able to recover the
% original text particularly accurately -- it's for the user to look at.
% Therefore, we just output a $|@|_{11}$ and use |\string| on the next
% token.  This must be sufficient, since we only allow multi-token command
% names if the first token is a letter (code~11).
%
%    \begin{macrocode}
\def\at@protect@string{@\string}
%    \end{macrocode}
%
% \end{macro}
%
% \begin{macro}{\at@protect@noexpand}
%
% This is a little more complex, since we're still expecting to be executed
% properly at some stage.  However, there's a cheeky dodge we can employ
% since the |\at| command is thoroughly robustified (or at least it will be
% by the time we've finished this).  All |\@unexpandable@protect| does
% is confer repeated robustness on a fragile command.  Since our command
% is robust, we don't need this and we can get away with just using a
% single |\noexpand|, both for the |\@at@| command and the following token
% (which we must robustify, because no-one else can do it for us -- if
% anyone tries, they end up using the |@\protect| command which is rather
% embarassing).
%
% I'll give the definition, and then examine how this expands in various
% cases.
%
%    \begin{macrocode}
\def\at@protect@noexpand{\noexpand\@at@ @\noexpand}
\def\@at@#1{\at}
%    \end{macrocode}
%
% A few points, before we go into the main examination of the protection.
% I've inserted a $|@|_{11}$ token, which is gobbled by |\@at@| when the
% thing is finally expanded fully.  This prevents following space tokens
% in an |\input| file from being swallowed because they follow a control
% sequence.  (I can't use the normal $|@|_{13}$ token, because when files
% like the |.aux| file are read in, |@| is given code~11 by
% |\makeatletter|.)
%
% \setbox0\hbox{|@at@|}
% Now for a description of why this works.  When |\at| is expanded, it works
% out that |\protect| is either |\noexpand| or |\@unexpandable@protect|, and
% becomes |\at@protect@noexpand|.  Because of the |\noexpand| tokens, this
% stops being expanded once it reaches $\fbox{\box0}\,|@|_{11}\,x$ (where
% $x$ is the token immediately following the $|@|_{13}$ character).  If this
% is expanded again, for example in another |\edef|, or in a |\write| or a
% |\mark|, the |\@at@| wakes up, gobbles the following |@| (whatever catcode
% it is -- there may be intervening |\write| and |\input| commands) and
% becomes |\at|, and the whole thing can start over again.
%
% \end{macro}
%
%
% \subsection{Enabling and disabling @-commands}
%
% \begin{macro}{\aton}
%
% We define the |\aton| command to enable all of our magic.  We store
% the old catcode in the |\atoff| command, make `@@' active, and make it
% do the stuff.
%
%    \begin{macrocode}
\def\aton{%
  \ifnum\catcode`\@=\active\else%
    \edef\atoff{\catcode`\noexpand\@\the\catcode`\@}%
    \catcode`\@\active%
    \lccode`\~`\@%
    \lowercase{\let~\at}%
  \fi%
}
%    \end{macrocode}
%
% \end{macro}
%
% \begin{macro}{\atoff}
%
% The |\atoff| command makes `@@' do the stuff it's meant to.  We remember
% the old catcode and revert to it.  This is largely unnecessary.
%
%    \begin{macrocode}
\def\atoff{\catcode`\@12}
%    \end{macrocode}
%
% \end{macro}
%
% \begin{macro}{\makeatother}
%
% Now we make our active `@@' the default outside of package files.
%
%    \begin{macrocode}
\let\makeatother\aton
%    \end{macrocode}
%
% \end{macro}
%
% And we must make sure that the user can use all of our nice commands.
% Once the document starts, we allow @-commands.
%
%    \begin{macrocode}
\AtBeginDocument{\aton}
%    \end{macrocode}
%
% \begin{macro}{\dospecials}
% \begin{macro}{\@sanitize}
%
% We must add the `@@' character to the various specials lists.
%
%    \begin{macrocode}
\expandafter\def\expandafter\dospecials\expandafter{\dospecials\do\@}
\expandafter\def\expandafter\@sanitize\expandafter{%
  \@sanitize\@makeother\@}
%    \end{macrocode}
%
% \end{macro}
% \end{macro}
%
% \subsection{Default @-commands}
%
% We define some trivial examples to get the user going.
%
%    \begin{macrocode}
\expandafter\chardef\csname at.@\endcsname=`\@
\atdef*#1*{\ifmmode\mathbf{#1}\else\textbf{#1}\fi}
\atdef/#1/{\ifmmode\mathit{#1}\else\emph{#1}\fi}
\atlet i=\index
\atdef I#1{#1\index{#1}}
%</package>
%    \end{macrocode}
%
% \hfill Mark Wooding, \today
%
% \Finale
%
\endinput

\chapter{Using the IDE}

The IDE (\textbf{I}ntegrated \textbf{D}evelopment \textbf{E}nvironment) 
provides a comfortable user interface to the compiler. The IDE is a textmode
application this allows the same look and feel for all supported
operating systems. Furthermore, it will be very familiar to all
Turbo Pascal users.
Currently is the IDE available for DOS, Win32 and Linux.

\section{First steps with the IDE}

The IDE is started by entering the command:
\begin{verbatim}
fp
\end{verbatim}
at the command line. Of course, you can also start the IDE
from a graphical user interface (under the Windows enviroment, you
can switch between windowed mode and full screen mode by pressing
Alt-Enter).

You can exit from the IDE by selecting \var{File|Exit} or by pressing
Alt-X.

\subsection{IDE Command line options}

When starting the IDE, command line options can be passed:
\begin{verbatim}
fp [-option] [-option] ... <file name> ...
\end{verbatim}

Option is one of the following switches (the option letters
aren't case sensitive):

\begin{description}
\item [-N] Only DOS: Don't use long file names. Since Win95 an interface
is provided to DOS applications to access long file names. The IDE uses
this interface by default to access files. Under certain circumstances, this
can lead to problems.
\item [-Cfilename] This option followed by (without spaces)
a filename uses the given file to read the options
\item [-R] After starting the IDE, it changes automatically to the directory
which was active when you left the IDE the last time.
\end{description}

Under DOS/Win32, \var{/} can be used instead of \var{\-} to pass a 
command line switch to the IDE

Furthermore, the files given at the command line are loaded into edit
windows automatically.

\section{Debugging programs with the IDE}

\section{Trouble shooting}

\section{Keyboard shortcuts}

A lot of keyboard shortcuts used by the IDE are compatible with the
good old WordStar and should be well known to Turbo Pascal users.

\begin{FPCltable}{p{5cm}ll}{Key shortcuts in the IDE}{shortcuts}
\hline \\
Command & Key shortcut & Alternative \\
\hline \\
Help & F1 & \\
Goto last help topic & Alt-F1 & \\
Search word at cursor position in help & Ctrl-F1 & \\
Help index & Shift-F1 & \\
Save file & F2 & Ctrl-K-S\\
Close active window & Alt-F3 & \\
Reset debugger/program & Ctrl-F2 & \\
Open file & F3 & \\
Display call stack & Ctrl-F3 & \\
Run til cursor & F4 & \\
Zomm/Unzoom window & F5 & \\
Switch to user screen & Alt-F5 & \\
Move/Zoom active window & Ctrl-F5 & \\
Switch to next window & F6 & \\
Switch to last window & Shift-F6 & \\
Trace into & F7 & \\
Add watch & Ctrl-F7 & \\
Step over & F8 & \\
Set breakpoint at current line & Ctrl-F8 & \\
Make & F9 & \\
Run & Ctrl-F9 & \\
Compile the active source file & Alt-F9 & \\
Menu & F10 & \\
Message & F11 & \\
Compiler messages & F12 & \\
List of windows & Alt-0 & \\
Exit IDE & Alt-X & \\
\hline \\
Char left & Arrow left & Ctrl-S \\
Char right & Arrow right & Ctrl-D \\
Line up & Arrow up & Ctrl-E \\
Line down & Arrow down & Ctrl-X \\
Word left & Ctrl-Arrow left & Ctrl-A \\
Word right & Ctrl-Arror right & Ctrl-F \\
Scroll one line up & Ctrl-W & \\
Scroll one line down & Ctrl-Z & \\
Page up & PageUp & Ctrl-R \\
Page down & PageDown & \\
Beginning of Line & Pos1 & Ctrl-Q-S \\
End of Line & End & Ctrl-Q-D \\
First line of window & Ctrl-Pos1 & Ctrl-Q-E \\
Last line of window & Ctrl-End & Ctrl-Q-X \\
First line of file & Ctrl-PageUp & Ctrl-Q-R \\
Last line of file & Ctrl-PageDown & Ctrl-Q-C \\
Last cursor position & Ctrl-Q-P & \\
\hline \\
Delete char & Del & Ctrl-G \\
Delete left char & Backspace & Ctrl-H \\
Delete line & Ctrl-Y & \\
Delete til end of line & Ctrl-Q-Y & \\
Delete word & Ctrl-T & \\
Insert line & Ctrl-N & \\
Toggle insert mode & Insert & Ctrl-V \\
\hline \\
Goto Beginning of selected text & Ctrl-Q-B & \\
Goto end of selected text & Ctrl-Q-K & \\
Mark beginning of selected text & Ctrl-K-B & \\
Mark end of selected text& Ctrl-K-K & \\
Remove selection & Ctrl-K-H & \\
Select current line & Ctrl-K-L & \\
Print selected text & Ctrl-K-P & \\
Select current word & Ctrl-K-T & \\
Delete selected text & Ctrl-Del & Ctrl-K-Y \\
Copy selected text to cursor position & Ctrl-K-C & \\
Move selected text to cursor position & Ctrl-K-V & \\
Copy selected text to clipboard & Ctrl-Ins & \\
Move selected text to the clipboard & Shift-Del & \\
Indent block one coloumn & Ctrl-K-I & \\
Unindent block one coloumn & Ctrl-K-U & \\
Insert text from clipboard & Shift-Insert & \\
Insert file & Ctrl-K-R & \\
Write selected text to file & Ctrl-K-W & \\
Extend selection one char to the left & Shift-Arrow left & \\
Extend selection one char to the right & Shift-Arrow right & \\
Extend selection to the beginning of the line & Shift-Pos1 & \\
Extend selection to the end of the line & Shift-End & \\
Extend selection to the same coloumn in the last row & Shift-Arrow up & \\
Extend selection to the same coloumn in the next row & Shift-Arrow down & \\
Extend selection to the end of the line & Shift-End & \\
Extend selection one word to the left & Ctrl-Shift-Arrow left & \\
Extend selection one word to the right & Ctrl-Shift-Arrow right & \\
Extend selection one page up & Shift-PageUp & \\
Extend selection one page down & Shift-PageDown & \\
Extend selection to the beginning of the file & Ctrl-Shift-Pos1 & Ctrl-Shift-PageUp \\
Extend selection to the end of the file & Ctrl-Shift-End & Ctrl-Shift-PageUp \\
\hline \\
Search & Ctrl-Q-F & \\
Search again & Ctrl-L & \\
Search and replace & Ctrl-Q-A & \\
Set mark & Ctrl-K-n (where n can be 0..9) & \\
Goto mark & Ctrl-Q-n (where n can be 0..9) & \\
Undo & Alt-Backspace & \\
\hline \\
\end{FPCltable}

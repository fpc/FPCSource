%
%   $Id$
%   This file is part of the FPC documentation.
%   Copyright (C) 1997, by Michael Van Canneyt
%
%   The FPC documentation is free text; you can redistribute it and/or
%   modify it under the terms of the GNU Library General Public License as
%   published by the Free Software Foundation; either version 2 of the
%   License, or (at your option) any later version.
%
%   The FPC Documentation is distributed in the hope that it will be useful,
%   but WITHOUT ANY WARRANTY; without even the implied warranty of
%   MERCHANTABILITY or FITNESS FOR A PARTICULAR PURPOSE.  See the GNU
%   Library General Public License for more details.
%
%   You should have received a copy of the GNU Library General Public
%   License along with the FPC documentation; see the file COPYING.LIB.  If not,
%   write to the Free Software Foundation, Inc., 59 Temple Place - Suite 330,
%   Boston, MA 02111-1307, USA. 
%
\documentclass{report}
\usepackage{a4}
\usepackage{html}
\makeindex
\latex{\usepackage{multicol}}
\latex{\usepackage{fpkman}}
\html{%
%   $Id$
%   This file is part of the FPC documentation.
%   Copyright (C) 1997, by Michael Van Canneyt
%
%   The FPC documentation is free text; you can redistribute it and/or
%   modify it under the terms of the GNU Library General Public License as
%   published by the Free Software Foundation; either version 2 of the
%   License, or (at your option) any later version.
%
%   The FPC Documentation is distributed in the hope that it will be useful,
%   but WITHOUT ANY WARRANTY; without even the implied warranty of
%   MERCHANTABILITY or FITNESS FOR A PARTICULAR PURPOSE.  See the GNU
%   Library General Public License for more details.
%
%   You should have received a copy of the GNU Library General Public
%   License along with the FPC documentation; see the file COPYING.LIB.  If not,
%   write to the Free Software Foundation, Inc., 59 Temple Place - Suite 330,
%   Boston, MA 02111-1307, USA. 
%
% Dummy
\newenvironment{FPKList}{\begin{description}}{\end{description}}
\newcommand{\functionl}[7]{
\subsection{#1}
\label{fu:#2}
\index{#1}
\subsubsection*{Declaration:}
\texttt {Function #1  #3  : #4;}
\subsubsection*{Description:}
#5
\subsubsection*{Errors:}
#6
\subsubsection*{See also:}
#7
}
\newcommand{\procedurel}[6]{
\subsection{#1}
\label{pro:#2}
\index{#1}
\subsubsection*{Declaration:}
\texttt {Procedure #1  #3 ;}
\subsubsection*{Description:}
#4
\subsubsection*{Errors:}
#5
\subsubsection*{See also:}
#6
}
\newcommand{\seefl}[2]{
\htmlref{#1}{fu:#2}
}
\newcommand{\seepl}[2]{
\htmlref{#1}{pro:#2}
}
%
% Now the ones without label.
%
\newcommand{\seef}[1]{\seefl{#1}{#1}}
\newcommand{\seep}[1]{\seepl{#1}{#1}}
%
\newcommand{\seet}[1]{
\htmlref{#1}{sec:types}
}
\newcommand{\seem}[2] {\texttt{#1} (#2) }
\newcommand{\var}[1]{\texttt {#1}}
\newcommand{\file}[1]{\textsf {#1}}
%
% procedures without args
%
\newcommand{\Procedurel}[5]{\procedurel{#1}{#2}{}{#3}{#4}{#5}}
\newcommand{\Functionl}[6]{\functionl{#1}{#2}{}{#3}{#4}{#5}{#6}}
%
% Procedures without labels.
%
\newcommand{\procedure}[5]{\procedurel{#1}{#1}{#2}{#3}{#4}{#5}}
\newcommand{\function}[6]{\functionl{#1}{#1}{#2}{#3}{#4}{#5}{#6}}
\newcommand{\Procedure}[4]{\procedure{#1}{}{#2}{#3}{#4}}
\newcommand{\Function}[5]{\function{#1}{}{#2}{#3}{#4}{#5}}
\newcommand{\linux}{\textsc{LinuX} }
\newcommand{\dos}  {\textsc{dos} }
\newcommand{\msdos}{\textsc{ms-dos} }
\newcommand{\ostwo}{\textsc{os/2} }
\newcommand{\windowsnt}{\textsc{WindowsNT} }
\newcommand{\docdescription}[1]{}
\newcommand{\docversion}[1]{}
\newcommand{\unitdescription}[1]{}
\newcommand{\unitversion}[1]{}
\newcommand{\fpk}{Free Pascal }
\newcommand{\gnu}{gnu }
%
% Useful references.
%
\newcommand{\progref}{\htmladdnormallink{Programmer's guide}{../prog/prog.html}\ }
\newcommand{\refref}{\htmladdnormallink{Reference guide}{../ref/ref.html}\ }
\newcommand{\userref}{\htmladdnormallink{Users' guide}{../user/user.html}\ }
\newcommand{\seecrt}{\htmladdnormallink{CRT}{../crt/crt.html}}
\newcommand{\seelinux}{\htmladdnormallink{Linux}{../linux/linux.html}}
\newcommand{\seestrings}{\htmladdnormallink{strings}{../strings/strings.html}}
\newcommand{\seedos}{\htmladdnormallink{DOS}{../dos/dos.html}}
\newcommand{\seegetopts}{\htmladdnormallink{getopts}{../getopts/getopts.html}}
\newcommand{\seeobjects}{\htmladdnormallink{objects}{../objects/objects.html}}
\newcommand{\seegraph}{\htmladdnormallink{graph}{../graph/graph.html}}
\newcommand{\seeprinter}{\htmladdnormallink{printer}{../printer/printer.html}}
\newcommand{\seego}{\htmladdnormallink{GO32}{../go32/go32.html}}
%
% Nice environments
%
% For Code examples (complete programs only)
\newenvironment{CodEx}{}{}
% For Tables.
\newenvironment{FPKtable}[2]{\begin{table}\caption{#2}\begin{center}\begin{tabular}{#1}}{\end{tabular}\end{center}\end{table}}
% The same, but with label in third argument (tab:#3)
\newenvironment{FPKltable}[3]{\begin{table}\caption{#2}\label{tab:#3}\begin{center}\begin{tabular}{#1}}{\end{tabular}\end{center}\end{table}}
%
% Commands to reference these things.
%
\newcommand{\seet}[1]{table (\ref{tab:#1}) }
\newcommand{\seec}[1]{chapter (\ref{ch:#1}) }
\newcommand{\sees}[1]{section (\ref{se:#1}) }
}
% define the version number here, and not in the fpk.sty !!! 
\newcommand{\fpkversion}{0.9.5}
\newcommand{\remark}[1]{\par$\rightarrow$\textbf{#1}\par}
% define many-used references.
%\newcommand{\progref}{\htmladdnormallink{Programmer's guide}{../prog/prog.html}\ }
%\newcommand{\refref}{\htmladdnormallink{Reference guide}{../ref/ref.html}\ }
%\newcommand{\seecrt}{\htmladdnormallink{CRT}{../crt/crt.html}}
%\newcommand{\seelinux}{\htmladdnormallink{Linux}{../linux/linux.html}}
%\newcommand{\seestrings}{\htmladdnormallink{strings}{../strings/strings.html}}
%\newcommand{\seedos}{\htmladdnormallink{DOS}{../dos/dos.html}}
%\newcommand{\seegetopts}{\htmladdnormallink{getopts}{../getopts/getopts.html}}
%\newcommand{\seeobjects}{\htmladdnormallink{objects}{../objects/objects.html}}
%\newcommand{\seegraph}{\htmladdnormallink{graph}{../graph/graph.html}}
%\newcommand{\seeprinter}{\htmladdnormallink{printer}{../printer/printer.html}}
%\newcommand{\seego}{\htmladdnormallink{GO32}{../go32/go32.html}}
%
\newcommand{\olabel}[1]{\label{option:#1}}
% We should change this to something better. See \seef etc.
\newcommand{\seeo}[1]{See \ref{option:#1}}
\begin{document}
\title{Free Pascal :\\ User's manual}
\docdescription{User's manual for \fpk, version \fpkversion}
\docversion{1.0}
\date{July 1997}
\author{Micha\"el Van Canneyt\\Florian Kl\"ampfl}
\maketitle
\tableofcontents
%%%%%%%%%%%%%%%%%%%%%%%%%%%%%%%%%%%%%%%%%%%%%%%%%%%%%%%%%%%%%%%%%%%%%
% Introduction
%%%%%%%%%%%%%%%%%%%%%%%%%%%%%%%%%%%%%%%%%%%%%%%%%%%%%%%%%%%%%%%%%%%%%
\chapter{Introduction}

%%%%%%%%%%%%%%%%%%%%%%%%%%%%%%%%%%%%%%%%%%%%%%%%%%%%%%%%%%%%%%%%%%%%%%%
% About this document
\section{About this document}
This is the user's manual for \fpk . It describes the installation and use of
the \fpk compiler on the different supported platforms. 
It does not attempt to give an exhaustive list of all supported commands,
nor a definition of the Pascal language. Look at the
\refref for these things. 
For a description of the
possibilities and the inner workings of the compiler, see the
\progref. In the appendices of this document you will find lists of 
reserved words and compiler error messages (with descriptions).

This document describes the compiler as it is/functions at the time of 
writing. Since the compiler is under continuous development, some of the
things described here may be outdated. In case of doubt, consult the
\file{README} files, distributed with the compiler. 
The \file{README} files are, in case of conflict with this manual,
 authoritative.


%%%%%%%%%%%%%%%%%%%%%%%%%%%%%%%%%%%%%%%%%%%%%%%%%%%%%%%%%%%%%%%%%%%%%%%
% About the compiler
\section{About the compiler}
\fpk is a 32-bit compiler for the i386 processor\footnote{Work is being done
on a port to ALPHA Architecture}. Currently, it supports 2 operating systems:
\begin{itemize}
\item \dos
\item \linux
\end{itemize}
and work is in progress to port it to other platforms (notably, \ostwo and
\windowsnt).

\fpk is designed to be, as much as possible, source compatible with 
Turbo Pascal 7.0 and Delphi II (although this goal is not yet attained), 
but it also enhances these languages with elements like function overloading.
And, unlike these ancestors, it supports multiple platforms.

It also differs from them in the sense that you cannot use compiled units
from one system for the other.

Also, at the time of writing, there is no Integrated Development Environment
(IDE) available for \fpk. This gap will, hopefully, be filled in the future.

\fpk consists of three parts :
\begin{enumerate}
\item The compiler program itself.
\item The Run-Time Library (RTL).
\item Utility programs and units.
\end{enumerate}

Of these you only need the first two, in order to be able to use the compiler.
In this document, we describe the use of the compiler. The RTL is described in the
\refref.

%%%%%%%%%%%%%%%%%%%%%%%%%%%%%%%%%%%%%%%%%%%%%%%%%%%%%%%%%%%%%%%%%%%%%%%
% Getting more information.
\section{Getting more information.}
If the documentation doesn't give an answer to your questions, 
you can obtain more information on the Internet, on the following addresses:
\begin{itemize}
\item \htmladdnormallink{http://tfdec1.fys.kuleuven.ac.be/\~ michael/fpk.html}
{http://tfdec1.fys.kuleuven.ac.be/~michael/fpk.html} contains information 
on the \linux port of the compiler. It contains also useful mail addresses and
links to other places.
\item \htmladdnormallink{http://www.brain.uni-freiburg.de/\~klaus/fpk-pas}
{http://www.brain.uni-freiburg.de/~klaus/fpk-pas} is the main \fpk information site. 
It also contains the instructions for inscribing to the \textit{mailing-list}, 
another useful source of information.
\end{itemize}
Both places can be used to download the \fpk distribution, although you can 
probably find them on other places also.

Finally, if you think something should be added to this manual 
(entirely possible), please do not hesitate and contact me at 
\htmladdnormallink{michael@tfdec1.fys.kuleuven.ac.be}{mailto:michael@tfdec1.fys.kuleuven.ac.be}
.

Let's get on with something useful.

%%%%%%%%%%%%%%%%%%%%%%%%%%%%%%%%%%%%%%%%%%%%%%%%%%%%%%%%%%%%%%%%%%%%%
% Installation
%%%%%%%%%%%%%%%%%%%%%%%%%%%%%%%%%%%%%%%%%%%%%%%%%%%%%%%%%%%%%%%%%%%%%

\chapter{Installing the compiler}

%%%%%%%%%%%%%%%%%%%%%%%%%%%%%%%%%%%%%%%%%%%%%%%%%%%%%%%%%%%%%%%%%%%%%%%
% Before Installation : Requirements
\section{Before Installation : Requirements}

%
%

% System requirements
\subsection{System requirements}
The compiler needs at least the following hardware:
\begin{enumerate}
\item An I386 or higher processor. A coprocessor is not required, although it
will slow down your program's performance if you do floating point calculations.
\item 4 Mb of free memory. Under \dos, if you use DPMI memory management,
such as under Windows, you will need at least 8 Mb.
\item At least 500 Kb. free disk space.
\end{enumerate}

%
%

% Software requirements
\subsection{Software requirements}

\subsubsection{Under DOS}
The \dos distribution contains all the files you need to run the compiler
and compile pascal programs.

\subsubsection{Under Linux}
Under \linux you need to have the following programs installed :
\begin{enumerate}
\item \gnu \file{as}, the \gnu assembler.
\item \gnu \file{ld}, the \gnu linker.
\item Optionally (but highly recommended) : \gnu \file{make}. For easy
recompiling of the compiler and Run-Time Library, this is needed.
\end{enumerate}
Other than that, \fpk should run on almost any \linux system.

%%%%%%%%%%%%%%%%%%%%%%%%%%%%%%%%%%%%%%%%%%%%%%%%%%%%%%%%%%%%%%%%%%%%%%%
% Installing the compiler.
\section{Installing the compiler.}
The installation of \fpk is easy, but is platform-dependent.
We discuss the process for each platform separately.


%
%

% Installing under DOS
\subsection{Installing under DOS}
\subsubsection{Mandatory installation steps.}
First, you must get the latest distribution files of \fpk. They come as zip
files, which you must unzip first. The distribution zip file contains an
installation program \file{INSTALL.EXE}. You must run this program to install
the compiler. It allows you to select:
\begin{itemize}
\item What components you wish to install. (e.g do you want the sources or
not, do you want Free Vision etc.)
\item Where you want to install (the default location is \verb|C:\PP|).
\end{itemize}
The installation program generates a batch file which sets some environment
variables : \verb|SET_PP.BAT|. This file is located in the directory where
you installed \fpk. The installation program doesn't modify the 
\file{AUTOEXEC.BAT}, since many people (including the authors of \fpk) 
don't like this.

You can choose to insert a call to this batch file in your \file{AUTOEXEC.BAT}
file, like this :
\begin{verbatim}
  CALL C:\PP\SET_PP.BAT
\end{verbatim}
(This is assuming that you installed \fpk in the default location.)
In order to run \fpk from any directory on your system, you must extend 
your path variable to contain the \verb|C:\PP\BIN| directory.
You can choose to do this in your \file{AUTOEXEC.BAT} file, but you can also
insert a statement in the \verb|SET_PP.BAT| file. Whatever the location you
choose, It should look something like this : 
\begin{verbatim}
  SET PATH=%PATH%;C:\PP\BIN
\end{verbatim}
(Again, assuming that you installed in the default location).
 
If you want to use the graphic drivers you must modify the
environment variable \var{GO32}. Instructions for doing this can be found
in the documentation of the Graph unit, at the InitGraph procedure.

\subsubsection{Optional Installation: The coprocessor emulation}
For people who have an older CPU type, without math coprocessor (i387)
it is necessary to install a coprocessor emulation, since \fpk uses the
coprocessor to do all floating point operations.

The installation of the coprocessor emulation is handled by the 
installation program (\file{INSTALL.EXE}). However,
the installation program has currently a bug: If you select the
coprocessor emulation the program ignores this and you must do
this by hand. You should change the \var{GO32} environment variable in
the \verb|SET_PP.BAT| file, as follows:
\begin{verbatim}
SET GO32=emu C:\PP\DRIVERS\EMU387 
\end{verbatim}


%
%

% Installing under Linux
\subsection{Installing under Linux}
\subsubsection{Mandatory installation steps.}
The \linux distribution of \fpk comes in two flavors:
\begin{itemize}
\item an \file{aout} version, and
\item an \file{ELF} version.
\end{itemize}
If you don't know which of these flavors you must use, contact you system
administrator, and he will tell you. When that doesn't get you further, try
the \file{ELF} distribution. \file{aout} systems are outdated, and may not be
supported any more in the future.

Both flavors are shipped in \file{tar} archive files.
This means that you should untar them, in some directory where you have
write permission, using the following command:
\begin{verbatim}
tar -xvf fpk.tar
\end{verbatim}
We supposed here that you downloaded the file \file{fpk.tar} somewhere
from the Internet.

When the files are untarred, you will be left with more archive files, and
two install programs: an installation shell script, and a X-windows
installation program. Both have the same functionality.
To install \fpk, all that you need to do now is give the following command:
\begin{verbatim}
./install.sh
\end{verbatim}
Or, if you have the XForms libraries, you can start the X-Windows based
program.
\begin{verbatim}
./fpkinstall
\end{verbatim}
And then you must answer some questions. They're very simple, they're
concerned with 2 things :
\begin{enumerate}
\item Places where you can install different things.
\item Deciding if you want to install certain components (such as sources
and demo programs).
\end{enumerate}
If you run the installation program/script  as \var{root}, you can just accept all installation
defaults. If you don't run as \var{root}, you must take care to supply the
installation program with directory names where you have write permission,
as it will attempt to create the directories you specify.
In principle, you can install it wherever you want, though.

Whatever the installation program you used, 
at the end of installation, the installation program will generate a
configuration file for the \fpk compiler which reflects the settings
that you chose. It will install this file in the \file{/etc} directory, (if
you are not installing as \var{root}, this will fail, and in the
directory where you installed the libraries.

If you want the \fpk compiler to use this configuration file, it must be
present in \file{/etc}, or you can set the environment variable
\var{PPC\_CONFIG\_PATH}. Under \file{csh}, you can do this by adding  a 
\begin{verbatim}
setenv PPC_CONFIG_PATH /usr/lib/ppc/0.9.1
\end{verbatim}
line to your \file{.login} file in your home directory. 
(see also the next section) 

\subsubsection{Optional configuration steps}
You may wish to set some environment variables. The \linux version of \fpk
recognizes the following variables :
\begin{itemize}
\item \verb|PPC_EXEC_PATH| contains the directory where '\file{as}' and
'\file{ld}' are. (default \file{/usr/bin})
\item \verb|PPC_GCCLIB_PATH| contains the directory where \file{libgcc.a} is (no default)
\item \verb|PPC_CONFIG_PATH| specifies an alternate path to find
\file{ppc386.cfg} (default \file{/etc})
\item \verb|PPC_ERROR_FILE|  specifies the path and name of the error-definition file. 
                  (default \file{/usr/lib/ppc/errorE.msg})
\end{itemize}

These locations are, however, set in the sample configuration file which is 
built at the end of the installation process, except for the
\verb|PPC_CONFIG_PATH| variable, which you must set if you didn't install
things in the default places.
\subsubsection{finally}
Also distributed in \fpk is a README file. It contains the latest
instructions for installing \fpk, and should always be read first.


%%%%%%%%%%%%%%%%%%%%%%%%%%%%%%%%%%%%%%%%%%%%%%%%%%%%%%%%%%%%%%%%%%%%%%%
% Testing the compiler
\section{Testing the compiler}
After the installation is completed and the environment variables are
set as described above, your first program can be compiled. 

Included in the \fpk distribution are some demonstration programs, 
showing what the compiler can do. 
You can test if the compiler functions correctly by trying to compile 
these programs.

The compiler is called
\begin{itemize}
\item \file{PPC386.EXE} under \dos, and 
\item \file{ppc386} under \linux
\end{itemize}
To compile a program (e.g \verb|demo\hello.pp|) simply type :
\begin{verbatim}
  ppc386 -a hello
\end{verbatim}
at the command prompt. The option -a is needed currently to call 
the external assembler. 
This needed option will disappear when the the internal assembler works stable.
but at the moment it is a mandatory option.

If you got no error messages, the compiler has generated an executable 
called \file{hello}  (no extension) under \linux, and a file \file{hello.exe}
under \dos. 

To execute the program, simply type :
\begin{verbatim}
  hello
\end{verbatim}
If all went well, you should see the following friendly greeting:
\begin{verbatim}
Hello world
\end{verbatim}
In the \dos case, this friendly greeting may be preceded by some ugly
message from the \file{GO32} extender program. This unfriendly behavior can 
be switched off by setting the \file{GO32} environment variable.

%%%%%%%%%%%%%%%%%%%%%%%%%%%%%%%%%%%%%%%%%%%%%%%%%%%%%%%%%%%%%%%%%%%%%
% Usage
%%%%%%%%%%%%%%%%%%%%%%%%%%%%%%%%%%%%%%%%%%%%%%%%%%%%%%%%%%%%%%%%%%%%%
\chapter{Compiler usage}
Here we describe the essentials to compile a program and a unit. 
We also describe how to make a stand-alone executable of the 
compiled program under \dos. For more advanced uses of the compiler, 
see the section on configuring the compiler, and the 
\progref.

The examples in this section suppose that you have a \file{ppc386.cfg} which
is set up correctly, and which contains at least the path setting for the
RTL units. In principle this file is generated by the installation program.
You may have to check that it is in the correct place (see section
\ref{se:config_file} for more information on this).


%%%%%%%%%%%%%%%%%%%%%%%%%%%%%%%%%%%%%%%%%%%%%%%%%%%%%%%%%%%%%%%%%%%%%%%
% Compiling a program
\section{Compiling a program}
Compiling a program is very simple. Assuming that you have a program source
in the file \file{prog.pp}, you can compile this with the following command:
\begin{verbatim}
  ppc386 [options] prog.pp
\end{verbatim}
The square brackets [] indicate that what is between them is optional. 

If your program file has the \file{.pp} or \file{.pas} extension, 
you can omit this on the command line, e.g. in the previous example you 
could have typed:
\begin{verbatim}
  ppc386 [options] prog
\end{verbatim}

If all went well, the compiler will produce an executable, or, for version 1
of the \dos extender, a file which can be converted to an executable.

Under \linux and version 2 of the \dos extender, the file you obtained is 
the executable. You can execute it straight away, you don't need to do 
anything else. Under \dos,
additional processing is required. See the section on creating an
executable.

You will notice that there are also other files in your directory, with
extensions \file{.o} and \file{.s}. These contain, respectively, 
the assembler sources and the object files for your program. You can
safely delete the assembler file, you don't need it any 
more\footnote{One day this will be done automatically.}. If you compiled a
program, you can delete the object file (\file{.o}), but not if you compiled
a unit. Then the object file contains the code of the unit, and will be
linked in any program that uses the unit you compiled, so you shpuldn't
remove it.


%%%%%%%%%%%%%%%%%%%%%%%%%%%%%%%%%%%%%%%%%%%%%%%%%%%%%%%%%%%%%%%%%%%%%%%
% Compiling a unit
\section{Compiling a unit}

Compiling a unit is not essentially different from compiling a program.
The difference is mainly that the linker isn't called in this case.

To compile a unit in the file \file{foo.pp}, just type :
\begin{verbatim}
  ppc386  foo
\end{verbatim}
Recall the remark about file extensions in the previous section.

When all went well, you will be left with 2 (two) unit files:
\begin{enumerate}
\item \file{foo.ppu} This is the file describing the unit you just
compiled.
\item \file{foo.o} This file contains the actual code of the unit.
This file will eventually end up in the executables.
\end{enumerate}
Both files are needed if you plan to use the unit for some programs. 
So don't delete them. If you want to distribute the unit, you must
provide both the \file{.ppu} and \file{.o} file. One is useless without the
other.

The file containing the assembler (extension \file{.s}) can safely be
deleted. You don't need it anymore.


%%%%%%%%%%%%%%%%%%%%%%%%%%%%%%%%%%%%%%%%%%%%%%%%%%%%%%%%%%%%%%%%%%%%%%%
% Creating an executable for GO32V1, PMODE/DJ targets
\section{Creating an executable for GO32V1, PMODE/DJ targets}

This section applies only to \dos users. \linux users can skip this
section (unless they're cross-compiling)


%
%

% GO32V1
\subsection{GO32V1}
When compiling under \dos, GO32V2 is the default target. However, if you use
go32V1 (using the \var{-TDOS} switch), the 
compilation process leaves you with a file which you cannot execute right away.
There are 2 things you can do when compiling has finished.

The first thing is to use the \dos extender from D.J. Delorie to execute
your program :
\begin{verbatim}
  go32 prog
\end{verbatim}
This is fine for testing, but if you want to use a program regularly, it
would be easier if you could just type the program name, i.e.
\begin{verbatim}
  prog
\end{verbatim}
This can be accomplished by making a \dos executable of your compiled program.
 
There two ways to create a \dos executable (under \dos only): 
\begin{enumerate}
\item if the \file{GO32.EXE} is already
installed on the computers where the program should run, you must
only copy a program called \file{STUB.EXE} at the begin of
the AOUT file. This is accomplished with the \file{AOUT2EXE.EXE} program. 
which comes with the compiler:
\begin{verbatim}
AOUT2EXE PROG
\end{verbatim}
and you get a \dos executable which loads the \file{GO32.EXE} automatically. 
the \file{GO32.EXE} executable must be in current directory or be 
in a directory in the \var{PATH} variable.
\item
The second way to create a \dos executable is to put 
\file{GO32.EXE} at the beginning of the \file{AOUT} file. To do this, at the
command prompt, type :
\begin{verbatim}
COPY /B GO32.EXE+PROG PROG.EXE
\end{verbatim}
(assuming \fpk created a file called \file{PROG}, of course.)
This becomes then a stand-alone executable for \dos, which doesn't need the
\file{GO32.EXE} on the machine where it should run.
\end{enumerate}

%
%

% PMODE/DJ
\subsection{PMODE/DJ}
You can also use the PMODE/DJ extender to run your \fpk applications.
To make an executable which works with the PMODE extender, you can simply
create an GO32V2 executable (the default), and then convert it to a PMODE
executable with the following two extra commands:
\begin{enumerate}
\item First, strip the GO32V2 header of the executable:
\begin{verbatim}
EXE2COFF PROG.EXE
\end{verbatim}
(we suppose that \file{PROG.EXE} is the program generated by the compilation
process.
\item Secondly, add the PMODE stub:
\begin{verbatim}
COPY /B PMODSTUB.EXE+PROG PROG.EXE
\end{verbatim}
If the \file{PMODSTUB.EXE} file isn't in your local directory, you need to
supply the whole path to it.
\end{enumerate}

That's it. No additional steps are needed to create a PMODE extender
executable.

Be aware, though, that the PMODE extender doesn't support virtual memory, so
if you're short on memory, you may run unto trouble. Also, officially there
is not support for the PMODE/DJ extender. It just happens that the compiler
and some of the programs it generates, run under this extender too.


%%%%%%%%%%%%%%%%%%%%%%%%%%%%%%%%%%%%%%%%%%%%%%%%%%%%%%%%%%%%%%%%%%%%%%%
% Reducing the size of your program
\section{Reducing the size of your program}

When you created your program, it is possible to reduce its size. This
is possible, because the compiler leaves a lot of information in the
program which, strictly speaking, isn't required for the execution of
it. The surplus of information can be removed with a small program
called \file{strip}. It comes with the \var{GO32} development
environment under \dos, and is standard on \linux machines where you can
do development. The usage is simple. Just type
\begin{verbatim}
strip prog
\end{verbatim}
On the command line, and the \file{strip} program will remove all unnecessary
information from your program. This can lead to size reductions of up to
30 \%.

You can use the \var{-Xs} switch to let the compiler do this stripping
automatically. Under \linux, you can set the \var{-k-s} option, which does
the same.

%%%%%%%%%%%%%%%%%%%%%%%%%%%%%%%%%%%%%%%%%%%%%%%%%%%%%%%%%%%%%%%%%%%%%
% Problems
%%%%%%%%%%%%%%%%%%%%%%%%%%%%%%%%%%%%%%%%%%%%%%%%%%%%%%%%%%%%%%%%%%%%%
\chapter{Compiling problems}

%%%%%%%%%%%%%%%%%%%%%%%%%%%%%%%%%%%%%%%%%%%%%%%%%%%%%%%%%%%%%%%%%%%%%%%
% General problems
\section{General problems}
\begin{itemize}
\item \textbf{IO-error -2 at ...} : Under \linux you can get this message at
compiler startup. It means typically that the compiler doesn't find the
error definitions file. You can correct this mistake with the \var{-Fr}
option under \linux. (\seeo{Fr})
\item \textbf {Error : File not found : xxx} This typically happens when
your unit path isn't set correctly. Remember that the compiler looks for
units only in the current directory, and in the directory where the compiler
itself is. If you want it to look somewhere else too, you must explicitly
tell it to do so using the \var{-Up} option (\seeo{Up}).
\end{itemize}

%%%%%%%%%%%%%%%%%%%%%%%%%%%%%%%%%%%%%%%%%%%%%%%%%%%%%%%%%%%%%%%%%%%%%%%
% Problems you may encounter under DOS
\section{Problems you may encounter under DOS}
\begin{itemize}
\item \textbf{No space in environment}.\\
An error message like this can occur, if you call 
\verb|SET_PP.BAT| in the \file{AUTOEXEC.BAT}.\\
To solve this problem, you must extend your environment memory.
To do this, search a line in the \file{CONFIG.SYS} like
\begin{verbatim}
SHELL=C:\DOS\COMMAND.COM
\end{verbatim}
and change it to the following: 
\begin{verbatim}
SHELL=C:\DOS\COMMAND.COM /E:1024
\end{verbatim}
You may just need to specify a higher value, if this parameter is already set.
\item \textbf{ Coprocessor missing}\\ 
If the compiler writes
a message that there is no coprocessor, install
the coprocessor emulation.
\item \textbf{Not enough DPMI memory}\\ 
If you want to use the compiler with \var{DPMI} you must have at least
7-8 MB free \var{DPMI} memory.
\end{itemize}



%%%%%%%%%%%%%%%%%%%%%%%%%%%%%%%%%%%%%%%%%%%%%%%%%%%%%%%%%%%%%%%%%%%%%
% Configuration.
%%%%%%%%%%%%%%%%%%%%%%%%%%%%%%%%%%%%%%%%%%%%%%%%%%%%%%%%%%%%%%%%%%%%%
\chapter{Compiler configuration}
The output of the compiler can be controlled in many ways. This can be done
essentially in two distinct ways:
\begin{itemize}
\item Using command-line options.
\item Using the configuration file: \file{ppc386.cfg}.
\end{itemize}
The compiler first reads the configuration file. Only then the command line
options are checked. This creates the possibility to set some basic options
in the configuration file, and at the same time you can still set some
specific options when compiling some unit or program. First we list the
command line options, and then we explain how to specify the command
line options in the configuration file. When reading this, keep in mind
that the options are case sensitive. While this is customary for \linux, it
isn't under \dos.
  

%%%%%%%%%%%%%%%%%%%%%%%%%%%%%%%%%%%%%%%%%%%%%%%%%%%%%%%%%%%%%%%%%%%%%%%
% Using the command-line options
\section{Using the command-line options}

The available options are listed by category:

%
%

% General options
\subsection{General options}
\begin{description}
\item[-h] if you specify this option, the compiler outputs a list of all options, 
and exits after that.
\olabel{h}
\item[-i] This option tells the compiler to print the copyright information.
\olabel{i}
\item[-l] This option tells the compiler to print the \fpk logo on standard
output. It also gives you the \fpk version number.
\olabel{l}
\item[-Lx] Set the language the compiler uses for its messages. 
\olabel{L}
\var{x} can be one of the following:
\begin{itemize}
\item \textbf{D} : Use German.
\item \textbf{E} : Use English.
\end{itemize}
\end{description}

%
%

% Options for getting feedback
\subsection{Options for getting feedback}
\begin{description}
\item[-qxx] This option tells the compiler to print on stdout some
information on what it is doing. xx can be one of the following:
\olabel{q}
\begin{itemize}
\item \textbf{x is empty} : Be quiet. Don't output anything
\item \textbf{-} : Give some information.
\item \textbf{--} : Give a lot of information.
\end{itemize}
This is an obsolete option, and is kept only for backwards
compatibility. It may disappear in the future.\\
\item[-vxxx] Be verbose. \var{xxx} is a combination of the following :
\olabel{v}
\begin{itemize}
\item \var{e} : Tells the compiler to show only errors. This option is on by default.
\item \var{i} : Tells the compiler to show some general information.
\item \var{w} : Tells the compiler to issue warnings.
\item \var{n} : Tells the compiler to issue notes.
\item \var{l} : Tells the compiler to show the line numbers as it processes a
file. Numbers are shown per 100.
\item \var{u} : Tells the compiler to print the names of the files it opens.
\item \var{t} : Tells the compiler to print the names of the files it tries
to open. 
\item \var{p} : Tells the compiler to print the names of procedures and
functions as it is processing them.
\item \var{c} : Tells the compiler to warn you when it processes a
conditional.
\item \var{m} : Tells the compiler to write which macros are defined.
\item \var{d} : Tells the compiler to write other debugging info.
\end{itemize}
\item[-w] When this option is specified, the compiler issues warning. To
turn off warnings, specify \var{-w-}\\
This option is obsolete. and is kept only for backwards compatibility. It
may disappear in the future.
\olabel{w}
\end{description}

%
%

% Options concerning files and directories
\subsection{Options concerning files and directories}
\begin{description}
\item [-exxx] (\linux only) \file{xxx} specifies the directory where the 
compiler can find the executables \file{as} (the assembler) and \file{ld} (the
compiler).
\olabel{e}
\item [-Fexxx] This option tells the compiler to write errors, etc. to 
the file in \file{xxx}.
\olabel{Fe} 
\item [-Fgxxx] (\linux only) \file{xxx} specifies the path where the compiler
can find the \gnu C library.
\olabel{Fg}
\item [-Frxxx] (\linux only) \file{xxx} specifies the path where the
compiler can find the error-definitions file.
\olabel{Fr}
\item [-Upxxx] \olabel{Up} Tells the compiler to add \file{xxx} to the path where to find
units. \\
By default, the compiler only searches for units in the current directory 
and the directory where the compiler itself resides. This option tells the
compiler also to look in the directory \file{xxx}.
\end{description}

% Options controlling the kind of output.
\subsection{Options controlling the kind of output.}
for more information on these options, see also \progref
\begin{description}
\item [-a] \olabel{a} Tells the compiler to generate an assembler source file, and to
call an external assembler (\file{as}) to assemble this file. The file will
not be deleted.
\item [-Axxx] \olabel{A}specifies what kind of assembler should be generated . Here
\var{xxx} is one of the following :
\begin{itemize}
\item \textbf{att} : AT\&T assembler.
\item \textbf{o} : A unix .o (object) file.
\item \textbf{obj} : A OMF file for using the NASM assembler.
\item \textbf{nasm} : a coff file using the NASM assembler.
\item \textbf{wasm} : An assembler file for the Microsoft/Borland/Watcom assembler.
\end{itemize}
\item [-Chxxx] \olabel {Ch} Reserves \var{xxx} bytes heap.
\item [-Ci] \olabel{Ci} Generate Input/output checking code.
\item [-Co] \olabel{Co} Generate Integer overflow checking code.
\item [-Cr] \olabel{Cr} Generate Range checking code.
\item [-Csxxx] \olabel{Cs} Set stack size to \var{xxx}. (\ostwo only).
\item [-dxxx] \olabel{d} Define the symbol name \var{xxx}. This can be used
to conditionally compile parts of your code.
\item [-g] \olabel{g} Generate debugging information for debugging with
\file{gdb}.
\item[-On] \olabel{O} optimize the compiler's output; \var{n} can have one
of the following values :
\begin{description}
\item[a] simple optimizations
\item[g] optimize for size
\item[G] optimize for time
\item[x] optimize maximum
\item[2] optimize for Pentium II (tm)
\item[3] optimize for i386
\item[4] optimize for i486
\item[5] optimize for Pentium (tm)
\item[6] optimizations for PentiumPro (tm)
\end{description}
The exact effect of these effects can be found in the appendices of the 
\progref.
\item [-s] \olabel{s} Tells the compiler not to call the assembler and linker.
You must specify also \var{-a} if you specify this.
\item[-Txxx] \olabel{T}Specifies the target operating system. \var{xxx} can be one of
the following:
\begin{itemize}
\item \textbf{DOS} : \dos and the DJ DELORIE extender.
\item \textbf{OS2} : OS/2 (2.x) (this is still under development).
\item \textbf{LINUX} : \linux.
\item \textbf{WIN32} : Windows 32 bit (this is still under development).
\item \textbf{GO32V2} : \dos and version 2 of the DJ DELORIE extender.
\end{itemize}
\item [-Xx] \olabel{X} executable options. This tells the compiler what
kind of \linux executable should be generated. the parameter \var{x}
can be one of the following:
\begin{itemize}
\item \textbf{e} : (\linux only) Create an \file{ELF} executable (default).
\item \textbf{c} : (\linux only) Link with the C library. You should only use this when
you start to port \fpk to another operating system.
\item \textbf{s} : (\dos only) Strip the symbols from the executable.
\end{itemize}
\end{description}

%
%

% Options concerning the sources (language options)
\subsection{Options concerning the sources (language options)}
for more information on these options, see also \progref
\begin{description}
\item [-Rxxx] \olabel{R} Specifies what assembler you use in your \var{asm} assembler code
blocks. Here \var{xxx} is one of the following:
\begin{description}
\item [att\ ] \var{asm} blocks contain AT\&T assembler.
\item [intel] \var{asm} blocks contain Intel assembler.
\item [direct] \var{asm} blocks should be copied as-is in the assembler
file.   
\end{description}
\item [-S2] \olabel{Stwo} Switch on Delphi 2 extensions.
\item [-Sann] \olabel{Sa} How severe should the compiler check your code ?
\var{nn} can be one of the following:
\begin{itemize}
\item \var{0} : Only ANSI Pascal expressions allowed.
\item \var{1} : Do not necessarily assign function results to variables.
\item \var{2} : Address operator \var{@} returns a typed pointer.
\item \var{4} : Assignment results are typed. (This allows constructs like
\var{a:=b:=0}. See also ...
\item \var{9} : Allows expressions with no side effect. \remark{Florian ???}
\end{itemize}
\item [-Sc] \olabel{Sc} Support C-style operators, i.e. \var{*=, +=, /= and
-=}.
\item [-Sg] \olabel{Sg} Support the \var{label} and \var{goto} commands.
\item [-Si] \olabel{Si} Support \var{C++} style INLINE.
\item [-Sm] \olabel{Sm} Support C-style macros.
\item [-So] \olabel{So} Try to be Borland TP compatible (no function
overloading etc.).
\item [-Ss] \olabel{Ss} The name of constructors must be \var{init}, and the
name of destructors should be \var{done}.
\item [-St] \olabel{St} Allow the \var{static} keyword in objects.
\item [-Un] \olabel{Un} Do not check the unit name. (Normally, the unit name
is the same as the filename. This option allows both to be different.)
\item [-Us] \olabel{Us} Compile a system unit. This option causes the
compiler to define only some very basic types.
\end{description}


%%%%%%%%%%%%%%%%%%%%%%%%%%%%%%%%%%%%%%%%%%%%%%%%%%%%%%%%%%%%%%%%%%%%%%%
% Using the configuration file
\section{Using the configuration file}
\label{se:config_file}
Using the configuration file \file{ppc386.cfg} is an alternative to command
line options. When a configuration file is found, it is read, and the lines
in it are treated like you typed them on the command line. They are treated
before the options that you type on the command line.

The compiler looks for the \file{ppc386.cfg} file in the following places :
\begin{enumerate}
\item The current directory.
\item Under \dos, the directory where the compiler is. Under \linux, 
       the compiler looks in the \file{/etc} directory, or, if specified,
the directory in the \var{PPC\_CONFIG\_PATH} environment variable.
\end{enumerate}
When the compiler has finished reading the configuration file, it continues
to treat the command line options.

One of the command-line options allows you to specify a second configuration
file: Specifying \file{@foo} on the command line will open file \file{foo},
and read further options from there. When the compiler has finished reading
this file, it continues to process the command line.

An important feature in the configuration file is that you can specify
sections. They behave much like conditional defines. 
Suppose the following configuration file (named \file{myconf})
\begin{verbatim}
-a
#section first
-Up/some_path
#section second
-Up/other_path.
\end{verbatim}
When you invoke the compiler as follows:
\begin{verbatim}
  ppc386 -dfirst @myconf foo.pp
\end{verbatim}
then the compiler will read the part of the configuration file coming before
the line containing \var{\#section second}. As a result the unit search path will be set
to \file{/some\_path}.
If, on the other hand, you invoke the compiler as 
\begin{verbatim}
  ppc386 -dsecond @myconf foo.pp
\end{verbatim}
Then the configuration file will be read as if the part between
\var{\#section first} and \var{\#section second} didn't exist, resulting
in a unit search path of \file{/other\_path}.
If you put a \var{\#section common} on a line, everything that follows this
keyword will be read, whatever the defined constants.

In short, the \var{\#define} keywords act as conditionals.


%%%%%%%%%%%%%%%%%%%%%%%%%%%%%%%%%%%%%%%%%%%%%%%%%%%%%%%%%%%%%%%%%%%%%
% Porting.
%%%%%%%%%%%%%%%%%%%%%%%%%%%%%%%%%%%%%%%%%%%%%%%%%%%%%%%%%%%%%%%%%%%%%

\chapter{Porting Turbo Pascal Code}

\fpk was designed to resemble Turbo Pascal as closely as possible. There
are, of course, restrictions. Some of these are due to the fact that \fpk is
a 32-bit compiler. Other restrictions result from the fact that \fpk works
on more than one operating system.

In general we can say that if you keep your program code close to ANSI
Pascal, you will have no problems porting from Turbo Pascal, or even Delphi, to
\fpk. To a large extent, the constructs defined by Turbo Pascal are
supported.

In the following sections we will list the Turbo Pascal constructs which are
not supported in \fpk, and we will list in what ways \fpk extends the Turbo
Pascal language.


%%%%%%%%%%%%%%%%%%%%%%%%%%%%%%%%%%%%%%%%%%%%%%%%%%%%%%%%%%%%%%%%%%%%%%%
% Things that will not work
\section{Things that will not work}
Here we give a list of things which are defined/allowed in Turbo Pascal, but
which are not supported by \fpk. Where possible, we indicate the reason. 
\begin{enumerate}
\item Parameter lists of previously defined functions and procedures must
match exactly. The reason for this is the function overloading mechanism of
\fpk. (however, \seeo{So})
\item \var {(* ... *)} as comment delimiters are not allowed in versions
older than 0.9.1. This can easily be remedied with a grown-up editor. 
\item The \var{MEM, MEMW, MEML} and \var{PORT} variables for memory and port
access are not available. This is due to the operating system. Under
\dos, the extender unit (\file {GO32.PPU} provides functions to remedy this.
\item \var{PROTECTED, PUBLIC, TRY, THROW, EXCEPTION} are reserved words.
This means you cannot create procedures or variables with the same name.
While they are not reserved words in Turbo Pascal, they are in Delphi.
\item The reserved words \var{FAR, NEAR} are ignored. This is
because \fpk is a 32 bit compiler, so they're obsolete.
\item \var{INTERRUPT} only will work on a DOS machine.
\item Boolean expressions are only evaluated until their result is completely
determined. The rest of the expression will be ignored.
\item At the moment of writing, the assembler syntax used in \fpk is \var{AT\&T}
assembler syntax. This is mainly because \fpk uses \gnu \var{as}.
\item Turbo Vision is not available.
\item The 'overlay' unit is not available. It also isn't necessary, since
\fpk is a 32 bit compiler, so program size shouldn't be a point.
\item There are more reserved words. (see appendix \ref{ch:reserved} for a
list of all reserved words.)
\item The command-line parameters of the compiler are different.
\item The compiler switches behave different.
\item Units are not binary compatible.
\end{enumerate}



%%%%%%%%%%%%%%%%%%%%%%%%%%%%%%%%%%%%%%%%%%%%%%%%%%%%%%%%%%%%%%%%%%%%%%%
% Things which are extra
\section{Things which are extra}
Here we give a list of things which are possible in \fpk, but which
didn't exist in Turbo Pascal or Delphi.
\begin{enumerate}
\item There are more reserved words. (see appendix \ref{ch:reserved} for a
list of all reserved words.)
\item Functions can also return complex types, such as records and arrays.
\item You can handle function results in the function itself, as a variable. 
Example
\begin{verbatim}
function a : longint;

begin
   a:=12;
   while a>4 do
     begin
        {...}
     end;
end;
\end{verbatim}
The example above would work with TP, but the compiler would assume
that the \var{a>4} is a recursive call. To do a recursive call in
this you must append \var{()} behind the function name: 
\begin{verbatim}
function a : longint;

begin
   a:=12;
   { this is the recursive call }
   if a()>4 then
     begin
        {...}
     end;
end;
\end{verbatim}
\item There is partial support of Delphi constructs. (see the \progref for
more information on this).
\item The \var{exit} call accepts a return value for functions.
\begin{verbatim}
function a : longint;

begin
   a:=12;
   if a>4 then
     begin
        exit(a*67); {function result upon exit is a*67 }
     end;
end;
\end{verbatim}
\item \fpk supports function overloading. That is, you can define many
functions with the same name, but with different arguments. For example:
\begin{verbatim}
procedure DoSomething (a : longint);
begin
{...}
end;

procedure DoSomething (a : real);
begin
{...}
end;
\end{verbatim} 
You can then call procedure \var{DoSomething} with an argument of type
\var{Longint} or \var{Real}.\\
This feature has the consequence that a previously declared function must
always be defined with the header completely the same:
\begin{verbatim}
procedure x (v : longint); forward;

{...} 

procedure x;{ This will overload the previously declared x}
begin
{...}
end;
\end{verbatim}
This construction will generate a compiler error, because the compiler
didn't find a definition of \var{procedure x (v : longint);}. Instead you
should define your procedure x as:
\begin{verbatim}
procedure x (v : longint);
{ This correctly defines the previously declared x}
begin
{...}
end;
\end{verbatim}
\end{enumerate}

%%%%%%%%%%%%%%%%%%%%%%%%%%%%%%%%%%%%%%%%%%%%%%%%%%%%%%%%%%%%%%%%%%%%%%%
% Turbo Pascal compatibility mode
\section{Turbo Pascal compatibility mode}
When you compile a program with the \var{-So} switch, the compiler will
attempt to mimic the Turbo Pascal compiler in the following ways:
\begin{itemize}
\item Assigning a procedural variable doesn't require a @ operator. One of
the differences between Turbo Pascal and \fpk is that the latter requires
you to specify an address operator when assigning a value to a procedural
variable. In Turbo Pascal compatibility mode, this is not required.
\item Procedure overloading is disabled.
\item Forward defined procedures don't need the full parameter list when
they are defined. Due to the procedure overloading feature of \fpk, you must
always specify the parameter list of a function when you define it, even
when it was declared earlier with \var{Forward}. In Turbo Pascal
compatibility mode, there is no function overloading, hence you can omit the
parameter list:
\begin{verbatim}
Procedure a (L : Longint); Forward;

...

Procedure a ; { No need to repeat the (L : Longint) }

begin
 ...
end;

\end{verbatim}
\item recursive function calls are handled dfferently. Consider the
following example :
\begin{verbatim}
Function expr : Longint;

begin
  ...
  Expr:=L:
  Writeln (Expr); 
  ...
end;
\end{verbatim}
In Turbo Pascal compatibility mode, the function will be called recursively 
when the \var{writeln} statement is processed. In \fpk, the function result
will be printed. In order to call the function recusively under \fpk, you
need to implement it as follows :
\begin{verbatim}
Function expr : Longint;

begin
  ...
  Expr:=L:
  Writeln (Expr()); 
  ...
end;
\end{verbatim}
\item Any text after the final \var{End.} statement is ignored. Normally,
this text is processed too.
\item You cannot assign procedural variables to void pointers.
\item The @ operator is typed when applied on procedures.
\item You cannot nest comments. 
\end{itemize}

%%%%%%%%%%%%%%%%%%%%%%%%%%%%%%%%%%%%%%%%%%%%%%%%%%%%%%%%%%%%%%%%%%%%%
% Utilities.
%%%%%%%%%%%%%%%%%%%%%%%%%%%%%%%%%%%%%%%%%%%%%%%%%%%%%%%%%%%%%%%%%%%%%

\chapter{Utilities and units that come with Free Pascal}
Besides the compiler and the Run-Time Library, \fpk comes with some utility
programs and units. Here we list these programs and units. 

%%%%%%%%%%%%%%%%%%%%%%%%%%%%%%%%%%%%%%%%%%%%%%%%%%%%%%%%%%%%%%%%%%%%%%%
% Supplied programs
\section{Supplied programs}

\begin{itemize}
\item \file{dumppu} is a program which shows the contents of a \fpk unit. It
comes in source form, and must be compiled before you can use it. Once
compiled, you can just issue the following command
\begin{verbatim}
  dumppu foo.ppu
\end{verbatim}
to display the contents of the \file{foo.ppu} unit.
\item Also distributed with Free Pascal comes a series of demonstration programs.
These programs have no other purpose than demonstrating the capabilities of
\fpk. They are located in the \file{demo} directory of the sources.
\item All example programs of the documentation are available. Check out the
directories that end on \file{ex} in the documentation sources. There you
wll find all example sources.
\end{itemize}


%%%%%%%%%%%%%%%%%%%%%%%%%%%%%%%%%%%%%%%%%%%%%%%%%%%%%%%%%%%%%%%%%%%%%%%
% Supplied units
\section{Supplied units}
Here we list the units that come with the \fpk distribution. Since there is
a difference in the supplied units per operating system, we list them
separately per system.

%
%

% Under DOS
\subsection{Under DOS}
\begin{itemize}
\item \seestrings\ This unit provides basic
string handling routines for the \var{pchar} type, comparable to similar
routines in standard \var{C} libraries.
\item \seeobjects\  This unit provides basic
routines for handling objects.
\item \seedos\ This unit provides basic routines for
accessing the operating system \dos. It provides almost the same
functionality as the Turbo Pascal unit. 
\item \seeprinter\  This unit provides all you
need for rudimentary access to the printer.
\item \seegetopts\ This unit gives you the
\gnu \var{getopts} command-line arguments  handling mechanism. 
It also supports long options.
\item \seecrt\ This unit provides basic screen
handling routines. It provides the same functionality  as the Turbo Pascal \var{CRT}
unit.
\item \seegraph\ This unit provides basic graphics
handling, with routines to draw lines on the screen, display texts etc. It
provides the same functions as the Turbo Pascal unit.
\item \seego\ This unit provides access to possibilities of the \var{GO32}
\dos extender.
\end{itemize}
\remark{Florian, I don't know the full list - let me know what is available}

%
%

% Under Linux
\subsection{Under Linux}
\begin{itemize}
\item \seestrings\ This unit provides basic
string handling routines for the \var{PChar} type, comparable to similar
routines in standard \var{C} libraries.
\item \seeobjects\ This unit provides basic
routines for handling objects.
\item \seecrt\ This unit provides basic screen
handling routines. It provides the same functionality Turbo Pascal \var{CRT}
unit. It works on any terminal which supports the \var{vt100} escape
sequences.
\item \seedos\ This unit provides an emulation of the
same unit under \dos. It is intended primarily for easy porting of Pascal
programs from \dos to \linux. For good performance, however, it is
recommended to use the \var{linux} unit.
\item \seelinux This unit provides access to the
\linux operating system. It provides most file and I/O handling routines
that you may need. It implements most of the standard \var{C} library constructs
that you will find on a Unix system. If you do a lot of disk/file
operations, the use of this unit is recommended over the one you use under
Dos.
\item \seeprinter\ This unit provides an
interface to the standard Unix printing mechanism.
\item \seegetopts This unit gives you the
\gnu \var{getopts} command-line arguments  handling mechanism. 
It also supports long options.
\end{itemize}

%%%%%%%%%%%%%%%%%%%%%%%%%%%%%%%%%%%%%%%%%%%%%%%%%%%%%%%%%%%%%%%%%%%%%
% Debugging
%%%%%%%%%%%%%%%%%%%%%%%%%%%%%%%%%%%%%%%%%%%%%%%%%%%%%%%%%%%%%%%%%%%%%

\chapter{Debugging your Programs}

\fpk supports debug information for the \gnu debugger \var{gdb}. 
This chapter describes shortly how to use this feature. It doesn't attempt
to describe completely the \gnu debugger, however.
For more information on the workings of the \gnu debugger, see the \var{gdb}
users' guide.


%%%%%%%%%%%%%%%%%%%%%%%%%%%%%%%%%%%%%%%%%%%%%%%%%%%%%%%%%%%%%%%%%%%%%%%
% Compiling your program with debugger support
\section{Compiling your program with debugger support}
First of all, you must be sure that the compiler is compiled with debugging
support. Unfortunately, there is no way to check this at run time, except by
trying to compile a program with debugging support.

To compile a program with debugging support, just specify the \var{-g}
option on the command-line, as follows:
\begin{verbatim}
ppc386 -g hello.pp
\end{verbatim}
This will generate debugging information in the executable from your
program. You will notice that the size of the executable increases
substantially because of this\footnote{A good reason not to include debug
information in an executable you plan to distribute.}. 

Note that the above will only generate debug information {\var for the code
that has been generated} when compiling \file{hello.pp}. This means that if
you used some units (the system unit, for instance) which were not compiled
with debugging support, no debugging support will be available for the code
in these units. 

There are 2 solutions for this problem. 
\begin{enumerate}
\item Recompile all units manually with the \var{-g} option. 
\item Specify the 'build' option (\var{-B}) when compiling with debugging
support. This will recompile all units, and insert debugging information in
each of the units.
\end{enumerate}
The second option may have undesirable side effects. It may be that some
units aren't found, or compile incorrectly due to missing conditionals,
etc..

If all went well, the executable now contains the necessary information with 
which you can debug it using \gnu \var{gdb}.


%%%%%%%%%%%%%%%%%%%%%%%%%%%%%%%%%%%%%%%%%%%%%%%%%%%%%%%%%%%%%%%%%%%%%%%
% Using \var{gdb
\section{Using \var{gdb} to debug your program}

To use gdb to debug your program, you can start the debugger, and give it as
an option the name of your program:
\begin{verbatim}
gdb hello
\end{verbatim}
This starts the debugger, and the debugger immediately loads your program
into memory, but it does not run the program yet. Instead, you are presented
with the following (more or less) message, followed by the \var{gdb} prompt
\var{'(gdb)'}:
\begin{verbatim}
GDB is free software and you are welcome to distribute copies of it
 under certain conditions; type "show copying" to see the conditions.
There is absolutely no warranty for GDB; type "show warranty" for details.
GDB 4.15.1 (i486-slackware-linux),
Copyright 1995 Free Software Foundation, Inc...
(gdb)
\end{verbatim}
To start the program you can use the \var{run} command. You can optionally 
specify command-line parameters, which will then be fed to your program, for
example:
\begin{verbatim}
(gdb) run -option -anotheroption needed_argument
\end{verbatim}
If your program runs without problems, \var{gdb} will inform you of this,
and return the exit code of your program. If the exit code was zero, then
the message \var{'Program exited normally'}.

If something went wrong (a segmentation fault or so), \var{gdb} will stop
the execution of your program, and inform you of this with an appropriate
message. You can then use the other \var{gdb} commands to see what happened.  
Alternatively, you can instruct \var{gdb} to stop at a certain point in your
program, with the \var{break} command.

Here is a short list of \var{gdb} commands, which you are likely to need when
debugging your program:
\begin{description}
\item [quit\ ] Exits the debugger.
\item [kill\ ] Stops a running program.
\item [help\ ] Gives help on all \var{gdb} commands.
\item [file\ ] Loads a new program into the debugger.
\item [directory\ ] Add a new directory to the search path for source
files.\\
{\em Remark:} My copy of gdb needs '.' to be added explicitly to the search
path, otherwise it doesn't find the sources.
\item [list\ ] Lists the program sources per 10 lines. As an option you can
specify a line number or function name.
\item [break\ ] Sets a breakpoint at a specified line or function
\item [awatch\ ] Sets a watch-point for an expression. A watch-point stops
execution of your program whenever the value of an expression is either 
read or written. 
\end{description}

for more information, see the \var{gdb} users' guide, or use the \var{'help'}
function in \var{gdb}.
%%%%%%%%%%%%%%%%%%%%%%%%%%%%%%%%%%%%%%%%%%%%%%%%%%%%%%%%%%%%%%%%%%%%%
% CGI.
%%%%%%%%%%%%%%%%%%%%%%%%%%%%%%%%%%%%%%%%%%%%%%%%%%%%%%%%%%%%%%%%%%%%%

\chapter{CGI programming in Free Pascal}
In these days of heavy WWW traffic on the Internet, CGI scripts have become
an important topic in computer programming. While CGI programming can be
done with almost any tool you wish, most languages aren't designed for it.
Perl may be a notable exception, but perl is an interpreted language, the
executable is quite big, and hence puts a big load on the server machine.

Because of its simple, almost intuitive, string handling and its easy syntax, 
Pascal is very well suited for CGI programming. Pascal allows you to quickly
produce some results, while giving you all the tools you need for more
complex programming. The basic RTL routines in principle are enough to get
the job done, but you can create, with relatively little effort, some units
which can be used as a base for more complex CGI programming.

That's why, in this chapter, we will discuss the basics of CGI in \fpk.
In the subsequent, we will assume that the server for which the programs are 
created, are based upon the NCSA \var{httpd} WWW server, as the examples
will be based upon the NCSA method of CGI programming\footnote{... and it's 
the only WWW-server I have to my disposition at the moment.}.

The two example programs in this chapter have been tested on the command line 
and worked, under the condition that no spaces were present in the name and 
value pairs provided to them.

%%%%%%%%%%%%%%%%%%%%%%%%%%%%%%%%%%%%%%%%%%%%%%%%%%%%%%%%%%%%%%%%%%%%%%%
% Getting your data
\section{Getting your data}
Your CGI program must react on data the user has filled in on the form which
your web-server gave him. The Web server takes the response on the form, and
feeds it to the CGI script.

There are essentially two ways of feeding the data to the CGI script. We will
discuss both.

%
%

% Data coming through standard input.
\subsection{Data coming through standard input.}
The first method of getting your data is through standard input. This method
is invoked when the form uses a form submission method of \var{POST}.
The web browser sets three environment variables \var{REQUEST\_METHOD},
\var{CONTENT\_TYPE} and \var{CONTENT\_LENGTH}. It feeds then the results of
the different fields through standard input to the CGI script.
All the Pascal program has to do is :
\begin{itemize}
\item Check the value of the \var{REQUEST\_METHOD} environment variable. The
\var{getenv} function will retrieve this value this for you.
\item Check the value of the \var{CONTENT\_TYPE} environment variable.
\item Read \var{CONTENT\_LENGTH} characters from standard input. \var{read
(c)} with \var{c} of type \var{char} will take care of that.
\end{itemize}
if you know that the request method will always be \var{POST}, and the
\var{CONTENT\_TYPE} will be correct, then you can skip the first two steps.
The third step can be done easier: read characters until you reach the
end-of-file marker of standard input.

The following example shows how this can be achieved:
\begin{verbatim}
program cgi_post;

uses dos; 

const max_data = 1000;

type datarec = record
  name,value : string;
  end;

var data : array[1..max_data] of datarec;
    i,nrdata : longint;
    c  : char;
    literal,aname : boolean;

begin
writeln ('Content-type: text/html');
writeln;
if getenv('REQUEST_METHOD')<>'POST' then
   begin
   writeln ('This script should be referenced with a METHOD of POST');
   write ('If you don''t understand this, see this ');
   write ('< A HREF="http://www.ncsa.uiuc.edu/SDG/Softare/Mosaic');
   writeln ('/Docs/fill-out-forms/overview.html">forms overview</A>.');
   halt(1);
   end;
if getenv('CONTENT_TYPE')<>'application/x-www-form-urlencoded' then
   begin
   writeln ('This script can only be used to decode form results');
   halt(1)
   end;
nrdata:=1;
aname:=true;
while not eof(input) do
  begin
  literal:=false;
  read(c);
  if c='\' then
     begin 
     literal:=true;
     read(c);
     end;
  if literal or ((c<>'=') and (c<>'&')) then
     with data[nrdata] do
        if aname then name:=name+c else value:=value+c
  else
     begin
     if c='&' then 
         begin
         inc (nrdata);
         aname:=true;
         end 
      else 
         aname:=false;
      end
  end;
writeln ('<H1>Form Results :</H1>');
writeln ('You submitted the following name/value pairs :');
writeln ('<UL>');
for i:=1 to nrdata do writeln ('<LI> ',data[i].name,' = ',data[i].value);
writeln ('</UL>');
end.
\end{verbatim}
While this program isn't shorter than the C program provided as an example
at NCSA, it doesn't need any other units. everythig is done using standard
Pascal procedures\footnote{actually, this program will give faulty results,
since spaces in the input are converted to plus signs by the web browser. 
The program doesn't check for this, but that is easy to change. 
The main concern here is to give the working principle.}.

Note that this program has a limitation: the length of names and values is
limited to 255 characters. This is due to the fact that strings in Pascal
have a maximal length of 255. It is of course easy to redefine the
\var{datarec} record in such a way that longer values are allowed.
In case you have to read the contents of a \var{TEXTAREA} form element,
this may be needed.


% Data passed through an environment variable
\subsection{Data passed through an environment variable}
If your form uses the \var{GET} method of passing it's data, the CGI script
needs to read the \var{QUERY\_STRING} environment variable to get it's data. 
Since this variable can, and probably will, be more than 255 characters long, 
you will not be able to use normal string methods, present in pascal. \fpk
implements the \var{pchar} type, which is a pointer to a null-terminated
array of characters.
And, fortunately, \fpk has a
\seestrings\  unit, which eases the use of the
\var{pchar} type.


The following example illustrates what to do in case of a method of \var{GET}
\begin{verbatim}
program cgi_get;

uses strings,linux; 

const max_data = 1000;

type datarec = record
  name,value : string;
  end;

var data : array[1..max_data] of datarec;
    i,nrdata : longint;
    p  : PChar;
    literal,aname : boolean;

begin
Writeln ('Content-type: text/html');
Writeln;
if StrComp(GetEnv('REQUEST_METHOD'),'POST')<>0 then
   begin
   Writeln ('This script should be referenced with a METHOD of GET');
   write ('If you don''t understand this, see this ');
   write ('< A HREF="http://www.ncsa.uiuc.edu/SDG/Softare/Mosaic');
   Writeln ('/Docs/fill-out-forms/overview.html">forms overview</A>.');
   halt(1);
   end;
p:=GetEnv('QUERY_STRING');
nrdata:=1;
aname:=true;
while p^<>#0  do
  begin
  literal:=false;
  if p^='\' then
     begin 
     literal:=true;
     inc(longint(p));
     end;
  if ((p^<>'=') and (p^<>'&')) or literal then
     with data[nrdata] do
        if aname then name:=name+p^ else value:=value+p^
  else
     begin
     if p^='&' then 
         begin
         inc (nrdata);
         aname:=true;
         end 
      else 
         aname:=false;
      end;
  inc(longint(p));
  end;
Writeln ('<H1>Form Results :</H1>');
Writeln ('You submitted the following name/value pairs :');
Writeln ('<UL>');
for i:=1 to nrdata do writeln ('<LI> ',data[i].name,' = ',data[i].value);
Writeln ('</UL>');
end.
\end{verbatim}
Although it may not be written in the most elegant way, this program does
the same thing as the previous one. It also suffers from the same drawback,
namely the limited length of the \var{value} field of the \var{datarec}.

This drawback can be remedied by redefining \var{datarec} as follows:
\begin{verbatim}
type datarec = record;
      name,value : pchar;
     end;
\end{verbatim}
and assigning at run time enough space to keep the contents of the value
field. This can be done with a
\begin{verbatim}
 getmem (data[nrdata].value,needed_number_of_bytes);
\end{verbatim}
call. After that you can do a 
\begin{verbatim}
strlcopy (data[nrdata].value,p,needed_number_of_bytes);
\end{verbatim}
to copy the data into place.

You may have noticed the following unorthodox call : 
\begin{verbatim}
inc(longint(p));
\end{verbatim}
\fpk doesn't give you pointer arithmetic as in C. However, \var{longints} and
\var{pointers} have the same length (namely 4 bytes). Doing a type-cast to a
\var{longint} allows you to do arithmetic on the \var{pointer}.

Note however, that this is a non-portable call. This may work on the I386
processor, but not on a ALPHA processor (where a pointer is 8 bytes long). 
This will be remedied in future releases of \fpk.


%%%%%%%%%%%%%%%%%%%%%%%%%%%%%%%%%%%%%%%%%%%%%%%%%%%%%%%%%%%%%%%%%%%%%%%
% Producing output
\section{Producing output}
The previous section concentrated mostly on getting input from the web
server. To send the reply to the server, you don't need to do anything
special.You just print your data on standard output, and the Web-server will
intercept this, and send your output to the WWW-client waiting for it.

You can print anything you want, the only thing you must take care of is
that you supply a \var{Contents-type} line, followed by an empty line, as
follows:
\begin{verbatim}
Writeln ('Content-type: text/html');
Writeln;
{ ...start output of the form... }

\end{verbatim}

And that's all there is to it !


%%%%%%%%%%%%%%%%%%%%%%%%%%%%%%%%%%%%%%%%%%%%%%%%%%%%%%%%%%%%%%%%%%%%%%%
% I'm under Windows, what now ?
\section{I'm under Windows, what now ?}
Under Windows the system of writing CGI scripts is totally different. If you
use \fpk under Windows then you also should be able to do CGI programming,
but the above instructions will not work. 

If some kind soul is willing to write a section on CGI programming under
Windows, I'd be willing to include it here.
\appendix


%%%%%%%%%%%%%%%%%%%%%%%%%%%%%%%%%%%%%%%%%%%%%%%%%%%%%%%%%%%%%%%%%%%%%
% APPENDIX A.
%%%%%%%%%%%%%%%%%%%%%%%%%%%%%%%%%%%%%%%%%%%%%%%%%%%%%%%%%%%%%%%%%%%%%

\chapter{Alphabetical listing of command-line options}
The following is alphabetical listing of all command-line options, as
generated by the compiler:
\begin{verbatim}
PPC386 [options] <inputfile> [options]
  + switch option on, - off
  with * marked options have no effect
  with ! marked options are only partial implemented
  -a     the compiler doesn't delete the generated assembler file
  -B     build
  -C     code generation options
           * -Ca 
           * -Ce 
                 
             -Chxxxx  xxxx bytes heap 
                 (must be less than 67107840 und greater than 1023
             -Ci IO-checking
             -Co check overflow of integer operations
             -Cr range checking
             -Csxxxx stack size (only OS/2)
             
  -dxxx  defines the symbol xxx
  -D     controlls the generation of DEF file (only OS/2)
             -Ddxxxx xxxx is the description
             -Do generate DEF file
             -Dw PM application
  -exxxx xxxx path to executables (only LINUX)
  -g     generate debugger informations
  -F     set file names and pathes
             -Fexxxx redirect error output to xxxx
             -Fgxxxx xxxxx search path for the GNU C lib (LINUX only)
             -Frxxxx xxxxx search path for the error message file (only LINUX)
  -L     set language
             -LD german
             -LE english
  -l     write logo
  -i     information
  -Ixxx  adds xxx to include path
  -n     don't read the default config file
  -oxxx  change the name of the executable produced to xxx
  -q-    write information when compiling  (obsolete, see -v)
  -S     syntax options
             -S2 switch some Delphi 2 extension on
             -Sa semantic check of expressions
                 a higher level includes the lower
                 -Sa0 only ANSI pascal expressions are allowed
                 -Sa1 functions results havn't to be assigned to variables
                 -Sa2 @-operator returns typed pointer
                 -Sa4 assigment results are typed (allows a:=b:=0)
                 -Sa9 allows expressions with no side effect
             -Sc supports operators like C (*=,+=,/= and -=)
             -Sg allows LABEL and GOTO
             -Si support C++ stlyed INLINE
             -Sm support macros like C (global)
             -So tries to be TP/BP 7.0 compatible
             -Ss the name of constructors must be init
                 the name of destructors must be done
             -St allows static keyword in objects
  -s     don't call assembler and linker (only with -a)
  -T     target operating system
             -TDOS DOS extender by DJ Delorie
             -TOS2 OS/2 2.x
             -TLINUX Linux
          !  -TWin32 Windows 32 Bit
             -TGO32V2 version 2 of DJ Delorie DOS extender
  -uxxx  undefines the symbol xxx
  -U     unit options
             -Un don't check the unit name
             -Us compile a system unit
             -Upxxxx adds xxxx to the unit path
  -vxxx  Be verbose. xxx is a combination of the following letters :
          e : Show errors (default)       i : Show general info
          w : Show warnings               l : Show linenumbers
          u : Show used files             t : Show tried files
          p : Show compiled procedures    c : Show conditionals
          d : Show debug info             m : Show defined macros
  -w-    turns warnings off (Obsolete, see -v) 
  -X     executable options
             -Xc link with the c library
             -Xe create ELF executable

Processor specific options:
  -A     output format
             -Aatt  AT&T assembler
             -Ao    coff file using GNU AS
             -Aobj  OMF file using NASM
             -Anasm coff file using NASM
             -Amasm assembler for the Microsoft/Borland/Watcom assembler
  -R     assembler reading style
             -Ratt  read AT&T style assembler
             -Rintel read Intel style assembler
             -Rdirect copy assembler text directly to assembler file
  -O     optimizations
             -Oa simple optimizations
             -Og optimize for size
             -OG optimize for time
             -Ox optimize maximum
             -O2 optimize for Pentium II (tm)
             -O3 optimize for i386
             -O4 optimize for i486
             -O5 optimize for Pentium (tm)
             -O6 optimizations for PentiumPro (tm)
  -h,-?  shows this help
\end{verbatim}


%%%%%%%%%%%%%%%%%%%%%%%%%%%%%%%%%%%%%%%%%%%%%%%%%%%%%%%%%%%%%%%%%%%%%
% APPENDIX B.
%%%%%%%%%%%%%%%%%%%%%%%%%%%%%%%%%%%%%%%%%%%%%%%%%%%%%%%%%%%%%%%%%%%%%

\chapter{Alphabetical list of reserved words}
\label{ch:reserved}
\latex{\begin{multicols}{3}}% \texttt
\begin{verbatim}
absolute
abstract
and
array
as
asm
assembler
begin
break
case
class
const
constructor
continue
destructor
dispose
div
do
downto
else
end
except
exit
export
exports
external
fail
false
far
file
finally
for
forward
function
goto
if
implementation
in
inherited
initialization
inline
interface
interrupt
is
label
library
mod
name
near
new
nil
not
object
of
on
operator
or
otherwise
packed
private
procedure
program
property
protected
public
raise
record
repeat
self
set
shl
shr
string
then
to
true
try
type
unit
until
uses
var
virtual
while
with
xor
\end{verbatim}
\latex{\end{multicols}}


%%%%%%%%%%%%%%%%%%%%%%%%%%%%%%%%%%%%%%%%%%%%%%%%%%%%%%%%%%%%%%%%%%%%%
% APPENDIX C.
%%%%%%%%%%%%%%%%%%%%%%%%%%%%%%%%%%%%%%%%%%%%%%%%%%%%%%%%%%%%%%%%%%%%%

\chapter{Compiler error messages}
\begin{description}
\item [unexpected end of file]
this typically happens in on of the following cases :
\begin{itemize}
\item The source file ends befor then final \var{end.} statement. This
happens mostly when the \var{begin} and \var{end} statements aren't
balanced;
\item An include file ends in the middle of a statement.
\item A comment wasn't closed.
\end{itemize}
\item [duplicate identifier:]
The identifier was already declared in the current scope.
\item [syntax error:]
An error against the Turbo Pascal language was encountered. This happens
typically when an illegal character is found in the sources file.
\item [out of memory]
The compiler doesn't have enough memory to compile your program. There are
several remedies for this:
\begin{itemize}
\item If you're using the build option of the compiler, try compiling the
different units manually.
\item If you're compiling a huge program, split it up in units, and compile
these separately.
\item If the previous two don't work, recompile the compiler with a bigger
heap (you can use the \var{-Ch} option for this, \seeo{Ch})
\end{itemize} 
\item [unknown identifier]
The identifier encountered hasn't been declared, or is used outside the
scope where it's defined.
\item [illegal character]
An illegal character was encountered in the input file. 
\item [source too long]
The compiler cannot cope with source files longer than (???)
\item [INLINE not supported (use option -Si for C++ styled inlining)]
You tried to compile a program with C++ style inlining, and forgot to
specify the \var{-Si} option (\seeo{Si}). The compiler doesn't support C++
styled inlining by default. 
\item [NEAR ignored]
This is a warning. \var{NEAR} is a construct for 8 or 16 bit programs. Since
the compile generates 32 bit programs, it ignores this directive.
\item [FAR ignored]
This is a warning. \var{FAR} is a construct for 8 or 16 bit programs. Since
the compile generates 32 bit programs, it ignores this directive.
\item [INTERRUPT ignored]
Interrupt procedures aren't possible on operating systems, other than DOS, 
it isn't allowed to take over an interrupt at the user level. (versions
older than 0.9.2 didn't have \var{INTERRUPT} support.
\item [private methods can't be VIRTUAL]
You declared a method in the private part of a object (class) as
\var{virtual}. This is not allowed. Private methods cannot be overridden
anyway.
\item [constructor can't be private or protected]
Constructors must be in the 'public' part of an object (class) declaration. 
\item [destructor can't be private or protected]
Destructors must be in the 'public' part of an object (class) declaration. 
\item [identifier not found]
\item [local class definitions are not allowed]
Classes must be defined globally.
\item [anonym class definitions are not allowed]
\item [type identifier expected]
The identifier is not a type, or you forgot to supply a type identifier.
\item [identifier already as type identifier declared]
You are trying to redefine a type.
\item [type identifier not defined]
The compiler encountered an unknown type.
\item [type mismatch]
This can happen in many cases:
\begin{itemize}
\item The variable you're assigning to is of a different type than the
expression in the assignment.
\item You are calling a function or procedure with parameters that are 
incompatible with the parameters in the function or procedure definition.
\end{itemize}
\item [statement expected]

\item [illegal integer constant]

\item [illegal expression]
\item [expression too complicated - FPU stack overflow]
Your expression is too long for the compiler. You should try dividing the
construct over multiple assignments.
\item [CONTINUE not possible]
You're trying to use \var{continue} outside a loop construction.
\item [BREAK not possible]
You're trying to use \var{break} outside a loop construction.
\item [exception handling needed to compile this (command line -Se)]
Older (less than 0.6.6) versions of \fpk only. Your statement needs exception handling. 
Exception handling isn't supported by default in the compiler. 
Use the \var{-Se} option to turn on exception handling.% (\seeo{Se})
\item [illegal qualifier]
One of the following is appending :
\begin{itemize}
\item You're trying to access a field of a variable that is not a record.
\item You're indexing a variable that is not an array.
\item You're dereferencing a variable that is not a pointer. 
\end{itemize}
\item [illegal count variable] The type of a \var{for} loop must be ordinal.

\item [ordinal type expected]
The expression must be of ordinal type (i.e. maximum a Longint)
\item [high range limit < low range limit]
You are declaring a subrange, and the lower limit is higher than the high
limit of the range.
\item [illegal unit identifier]
The name of the unit doesn't match the file name.
\item [unknown format of unit file]
The unit the compiler is trying to read is corrupted, or generated with a
newer version of the compiler.
\item [error when reading unit]
The unit the compiler is trying to read is corrupted, or generated with a
newer version of the compiler.
\item [circular unit use]
Two units are using each other in the interface part. This is only allowed
in the implementation part.
\item [too many units]
\fpk has a limit of 1024 units in a program. You can change this behavior
by changing the \var{maxunits} constant in the \file{files.pas} file of the
compiler, and recompiling the compiler. 
\item [illegal char constant]
\item [overloaded identifier isn't a function identifier]
\item [overloaded functions have the same parameter list]
You're declaring overloaded functions, but with the same parameter list.
Overloaded function must have at least 1 different parameter in their
declaration.
\item [illegal parameter list]
You are calling a function with parameters that are of a different type than
the declared parameters of the function.
\item [can't determine which overloaded function to call]
You're calling overloaded functions with a parameter that doesn't correspond
to any of the declared function parameter lists. e.g. when you have declared
a function with parameters \var{word} and \var{longint}, and then you call
it with a parameter which is of type \var{integer}.
\item [exception handling not used, however needed by function]
This message is no longer used.
\item [forward declaration not solved:]
This can happen in two cases:
\begin{itemize}
\item This happens when you declare a function (in the \var{interface} part, or
with a \var{forward} directive, but do not implement it.
\item You reference a type which isn't declared in the current \var{type}
block.
\end{itemize}
\item [input file not found]
\fpk cannot find the program or unit source file, or the included file isn't
found.
\item [function header doesn't match the forward declaration]
You declared the function in the \var{interface} part, or with the
\var{forward} directive, but define it with a different parameter list. 
\item [unknown field identifier]
The field doesn't exist in the record definition.
\item [parameter list size exceeds 65535 bytes]
The I386 processor limits the parameter list to 65535 bytes (the \var{RET}
instruction causes this)
\item [function nesting > 31]
You can nest function definitions only 31 times. 
\item [illegal compiler switch]
You included a compiler switch (i.e. \var{\{\$... \}}) which the compiler
doesn't know.
\item [can't open include file]
You want to include (i.e \var{\{\$i file\}}) which the compiler doesn't
find. Check if the filename is correct.
\item [record or class type expected]
The variable or expression isn't of the type \var{record} or \var{class}.
\item [not found:]
An unknown symbol was encountered.
\item [only values can be jumped over in enumeration types]
\fpk allows enumeration constructions as in C. Given the following
declaration two declarations:
\begin{verbatim}
type a = (A_A,A_B,A_E=:6,A_UAS:=200);
type a = (A_A,A_B,A_E=:6,A_UAS:=4);
\end{verbatim}
The second declaration would produce an error. \var{A\_UAS} needs to have a
value higher than \var{A\_E}, i.e. at least 7.
\item [pointer type expected]
The variable or expression isn't of the type \var{pointer}.
\item [unit is compiled for another operating system]
The unit was compiled with a different target than the target for which
you're compiling now. (see the option \var{-T} \seeo{T}).
\item [typed constants of classes are not allowed]
You cannot declare a constant of type class or object.
\item [duplicate case label]
You are specifying the same label 2 times in a \var{case} statement.
\item [range check error while Eva luting constants]
The constants are out of their allowed range.
\item [illegal type conversion]
When doing a type-cast, you must take care that the sizes of the variable and
the destination type are the same. 
\item [class type expected]
The variable of expression isn't of the type \var{class}.
\item [functions variables of overloaded functions are not allowed]
You are trying to assign an overloaded function to a procedural variable.
This isn't allowed.
\item [can't open assembler output file]
The assembler output file cannot be opened. This can have many causes, but
'disk full' is a reasonable guess.
\item [string length must be a value from 1 to 255]
The length of a string in Pascal is limited to 255 characters. You are
trying to declare a string with length greater than 255.
\item [class identifier expected]
The variable isn't of type \var{class}.
\item [method identifier expected]
This identifier is not a method.
\item [function header doesn't match any method of this class]
You are defining a function as a class method, but no such function was
declared in the class.
\item [use extended syntax of DISPOSE and NEW to generate instances of classes]
If you have a pointer \var{a} to a class type, then the statement
\var{new(a)} will not initialize the class (i.e. the constructor isn't
called), although space will be allocated. you should issue the
\var{new(a,init)} statement. This will allocate space, and call the
constructor of the class.
\item [file types must be var parameters]
You cannot specify files as value parameters, i.e. they must always be
declared \var{var} parameters.
\item [string constant exceeds line end]
You forgot probably to include the closing ' in a string. 
\item [illegal version of the unit:]
This unit was compiled with an earlier version of \fpk.
\item [illegal floating point constant]
\item [destructors can't have parameters]
You are declaring a destructor with a parameter list. Destructor methods
cannot have parameters.
\item [FAIL can be used in constructors only]
You are using the \var{FAIl} instruction outside a constructor method.
\item [records fields can be aligned to 1,2 or 4 bytes only]
You are specifying the \var{\{\$PACKRECORDS n\} } with an illegal value for
\var{n}. Only 1,2 or 4 are valid in this case.
\item [too many \$ENDIFs or \$ELSEs]
Your \var{\{\$IFDEF ..\}} and {\{\$ENDIF\}} statements aren't balanced.
\item [\$ENDIF expected]
Your \var{\{\$IFDEF ..\}} and {\{\$ENDIF\}} statements aren't balanced.
\item [illegal call by reference parameters]
\item [can't generate DEF file]
\ostwo only. The DEF file cannot be generated.
\item [all overloaded methods must be virtual if one is virtual:]
If you declare overloaded methods in a class, then they should either all be
virtual, or none. You cannot mix them.
\item [overloaded methods which are virtual must have the same return type:]
If you declare virtual overloaded methods in a class definition, they must
have the same return type.
\item [all overloaded virtual methods must support exceptions if one support exceptions:]
If you declare overloaded virtual methods in a class, then they should either 
all support exceptions, or none. You cannot mix them.
\item [EXPORT declared functions can't be called]
You are trying to call a procedure you declared as \var{export}. Due to the
different calling scheme of \fpk and C, you cannot call such a function from
within your Pascal program. 
\item [EXPORT declared functions can't be nested]
You cannot declare a function or procedure within a function or procedure
that was declared as an export procedure.
\item [methods can't be EXPORTed]
You cannot declare a procedure that is a method for an object as
\var{export}ed. That is, you methods cannot be called from a C program.
\item [SELF is allowed in methods only]
You are trying to use the \var{self} parameter outside an object's method.
Only methods get passed the \var{self} parameters.
\item [call by var parameters have to match exactly]
When calling a function declared with \var{var} parameters, the variables in
the function call must be of exactly the same type. There is no automatic
type conversion. 
\item [type identifiers are not allowed in this context]
\item [class identifier expected]
The variable isn't of type \var{class}.
\item [class isn't a super class of the current class]
When calling inherited methods, you are trying to call a method of a strange
class. You can only call an inherited method of a parent class.
\item [methods can be only in other methods called direct with type identifier of the class]
A construction like \var{sometype.somemethod} is only allowed in a method.
\item [illegal INHERITED: class has no super class]
You specified an \var{INHERITED} keyword in a method of a class which has no
parent class, i.e. which isn't derived from another class.
\item [illegal type: pointer to class expected]
You specified an illegal type.
\item [possible illegal call of constructor or destructor (doesn't match to this context)]
\item [class should have one destructor only]
You can declare only one destructor for a class.
\item [expression must be constructor call]
When using the extended syntax of \var{new}, you must specify the constructor
method of the class you are trying to create. The procedure you specified
is not a constructor.
\item [identifier idents no member]
When using the extended syntax of \var{new}, you must specify the constructor
method of the class you are trying to create. The procedure you specified
does not exist.
\item [expression must be destructor call]
When using the extended syntax of \var{dispose}, you must specify the
destructor method of the class you are trying to dispose of. 
The procedure you specified is not a destructor.

\item [assembler: illegal constant]
\item [illegal type declaration of set elements]
\item [illegal expression in set constructor]
\item [type conflict between set elements]
You are specifying elements of a different type for a set.
\item [illegal use of ':']
\item [expression type must be class or record type]
The expression isn't of type class or record.
\item [the operator / isn't defined for integer, the result will be real, use DIV instead]
When using the '/' operator in \fpk the result will be of type real, when
used with integers.
\item [can't write PPU file]
There is a problem when writing to the unit file.
\item [illegal order of record elements]
When declaring a constant record, you specified the fields in the wrong
order.
\item [the name of constructors must be INIT]
You are declaring a constructor with a name which isn't \var{init}, and the
\var{-Ss} switch is in effect. See the \var{-Ss} switch (\seeo{Ss}). 
\item [the name of constructors must be DONE]
You are declaring a constructor with a name which isn't \var{done}, and the
\var{-Ss} switch is in effect. See the \var{-Ss} switch (\seeo{Ss}). 
\item [set element type mismatch]
The type of the element doesn't equal the set type.
\item [illegal label declaration]
\item [label not found]
A \var{goto label} was encountered, but the label isn't declared. 
\item [GOTO and LABEL are not supported (use command line switch -Sg)]
You must compile a program which has \var{label}s and \var{goto} statements 
with the  \var{-Sg} switch. By default, \var{label} and \var{goto} aren't
supported.
\item [set expected]
The variable or expression isn't of type \var{set}.
\item [identifier isn't a label]
The identifier specified after the \var{goto} isn't of type label.
\item [label already defined]
You're attempting to define a label two times. (i.e. you put the same label
on two different places.)
\item [label isn't defined:]
A label was declared, but not defined.
\item [constructors and destructors must be methods]
You're declaring a procedure as destructor or constructor, when the
procedure isn't a class method.
\item [error when assembling]
An error occurred when assembling. This can have many causes.
\item [identifier not used:]
This is a warning. The identifier was declared (locally or globally) but
wasn't used (locally or globally).
\item [functions with void return value can't return any value]
In \fpk, you can specify a return value for a function when using 
the \var{exit} statement. This error occurs when you try to do this with a
procedure. Procedures  cannot return a value.
\item [Hmmm..., this code can't be much efficient]
You construction seems dubious to the compiler.
\item [unreachable code]
You specified a loop which will never be executed. Example:
\begin{verbatim}
while false do
  begin
  {.. code ...}
  end;
\end{verbatim}
\item [This overloaded function can't be local (must be exported)]
You are defining a overloaded function in the implementation part of a unit,
but there is no corresponding declaration in the interface part of the unit.
\item [It's not possible to overload this operator]
You are trying to overload an operator which cannot be overloaded.
\item [Abstract methods can't be called direct]
\fpk understands the \var{abstract} keyword.
\item [the mix of CLASSES and OBJECTS are not allowed]
You cannot use \var{object} and \var{class} intertwined.
\item [macro buffer overflow while reading or expanding a macro]
Your macro or it's result  was too long for the compiler.
\item [keyword redefined as macro has no effect]
You cannot redefine keywords with macros.
\item [extension of macros exceeds a deep of 16,\\ perhaps there is a recursive macro definition (crashes the compiler)]
When expanding a macro macros have been nested to a level of 16.
\item [ENDIF without IF(N)DEF]
Your code contains more \var{\{\$ENDIF\}} than \var{\{\$IF(N)DEF\}}
statements.
\item [user defined:]
A user defined warning occurred. see also the \progref
\item [linker: Duplicate symbol:]
Two global symbols in the code have the same name.
\item [linker: Error while reading object file]
The linker couldn't read the object file (the assembled file).
\item [linker: object file not found]
The linker didn't find the object file (the assembled file).
\item [linker: illegal magic number in file:]
The linker cannot determine the type of a file it wants to link in. The type
of a link file is specified using a magic number, which is some pre-defined
constant, unique for each system.
\item [The extended syntax of new or dispose isn't allowed for a class]
You cannot generate an instance of a class with the extended syntax of 
\var{new}. The constructor must be used for that. For the same reason, you
cannot call \var{Dispose} to de-allocate an instance of a class, the
destructor must be used for that.
\item [To generate an instance of a class or an object with an abstract method isn't allowed]
You are trying to generate an instance of a class which has an abstract
method that wasn't overridden.
\item [Only virtual methods can be abstract]
You are declaring a method as abstract, when it isn't declared to be
virtual.
\item [Abstract methods shouldn't have any definition (with function body)]
Abstract methods can only be declared, you cannot implement them. They
should be overridden by a descendant class.
\item [can't call the assembler]
An error occurred when calling the assembler.
\item [can't call o2obj]
An error occurred when calling the \var{o} to \var{obj} conversion program.
\item [asm syntax error]
There is an error in the assembly language.
\item [register name expected]
There is an error in the assembly language. The assembler expected a
register and got something else.
\item [asm size mismatch]
There is an error in the assembly language. The sizes of operands and
registers don't match.
\item [no instr match,]
There is an error in the assembly language. An unknown instruction was
encountered.
\item [can't compile unit:]
When trying to do a build, the compiler cannot compile one of the units.
\item [Re-raise isn't possible there]
You are trying to raise an exception where it isn't allowed. You can only
raise exceptions in an \var{except} block.
\end{description}

\chapter{Run time errors}
The \fpk Run-tim library generates the following errors at run-time
\footnote{The \linux port will generate only a subset of these.}:

\begin{description}
\item [1  Invalid function number]
You tried to call a \dos function which doesn't exist.
\item [2  File not found]
You can get this error when you tried to do an operation on a file which
doesn't exist.
\item [3  Path not found]
You can get this error when you tried to do an operation on a file which
doesn't exist, or when you try to change to, or remove a directory that doesn't exist,
or try to make a subdirectory  of a subdirectory that doesn't exist.
\item [4  Too many open files]
When attempting to open a file for reading or writing, you can get this
error when your program has too many open files.
\item [5  File access denied]
You don't have access to the specified file.
\item [6  Invalid file handle]
If this happens, the file variable you are using is trashed; it
indicates that your memory is corrupted.
\item [12  Invalid file access code]
This will happen if you do a reset or rewrite of a file when \var{FileMode}
is invalid. 
\item [15  Invalid drive number]
The number given to the Getdir function specifies a non-existent disk.
\item [16  Cannot remove current directory]
You get this if you try to remove the current diirectory.
\item [17  Cannot rename across drives]
You cannot rename a file such that it would end up on another disk or
partition.
\item [100  Disk read error]
\dos only. An error occurred when reading from disk. Typically when you try
to read past the end of a file.
\item [101  Disk write error]
\dos only. Reported when the disk is full, and you're trying to write to it.
\item [102  File not assigned]
This is reported by Reset, Rewrite, Append, Rename and Erase, if you call
them with an unassigne function as a parameter.
\item [103  File not open]
Reported by the following functions : Close , Read, Write, Seek,
EOf, FilePos, FileSize, Flush, BlockRead, and BlockWrite if the file isn't
open.
\item [104  File not open for input]
Reported by Read, BlockRead, Eof, Eoln, SeekEof or SeekEoln if the file
isn't opened with Reset.
\item [105  File not open for output]
Reported by write if a text file isn't opened with Rewrite.
\item [106  Invalid numeric format]
Reported when a non-numerice value is read from a text file, when a numeric
value was expected.
\item [150  Disk is write-protected]
(Critical error, \dos only.) 
\item [151  Bad drive request struct length]
(Critical error, \dos only.) 
\item [152  Drive not ready]
(Critical error, \dos only.) 
\item [154  CRC error in data]
(Critical error, \dos only.) 
\item [156  Disk seek error]
(Critical error, \dos only.) 
\item [157  Unknown media type]
(Critical error, \dos only.) 
\item [158  Sector Not Found]
(Critical error, \dos only.) 
\item [159  Printer out of paper]
(Critical error, \dos only.) 
\item [160  Device write fault]
(Critical error, \dos only.) 
\item [161  Device read fault]
(Critical error, \dos only.) 
\item [162  Hardware failure]
(Critical error, \dos only.) 
\item [200  Division by zero]
You are dividing a number by zero.
\item [201  Range check error]
If you compiled your program with range checking on, then you can get this
error in the following cases:
\begin{enumerate}
\item An array was accessed with an index outside its declared range.
\item You're trying to assign a value to a variable outside its range (for
instance a enumerated type).
\end{enumerate} 
\item [202  Stack overflow error]
The stack has grown beyond itss maximum size. This error can easily occur if
you have recursive functions. 
\item [203  Heap overflow error]
The heap has grown beyond its boundaries, ad you are rying to get more
memory. Please note that \fpk provides a growing heap, i.e. the heap will
try to allocate more memory if needed. However, if the heap has reached the
maximum size allowed by the operating system or hardware, then you will get
this error.
\item [204  Invalid pointer operation]
This you will get if you call dispose or Freemem with an invalid pointer
(notably, \var{Nil})
\item [205  Floating point overflow]
You are trying to use or produce too large real numbers. 
\item [206  Floating point underflow]
You are trying to use or produce too small real numbers. 
\item [207  Invalid floating point operation]
Can occur if you try to calculate the square root or logarithm of a negative
number.
\item [210  Object not initialized]
When compiled with range checking on, a program will report this error if
you call a virtal method without having initialized the VMT.
\item [211  Call to abstract method]
Your program tried to execute an abstract virtual method. Abstract methods
should be overridden, and the overriding method should be called.
\item [212  Stream registration error]
This occurs when an invalid type is registered in the objects unit.
\item [213  Collection index out of range]
You are trying to access a collection item with an invalid index.
(objects unit) 
\item [214  Collection overflow error]
The collection has reached its maximal size, and you are trying to add
another element. (objects unit)
\item [216  General Protection fault]
You are trying to access memory outside your appointed memory.
\end{description}
\end{document}
